In this section we describe the categorical semantics of
HoTT via Natural Models.
This will not be a detailed account of the syntax of HoTT,
but will be a detailed account of what is needed to interpret such syntax.
It will follow \cite{awodey2017naturalmodelshomotopytype},
but with a more compact description of identity types
using the technology of polynomial endofunctors,
and a universe of small types.

\medskip

\begin{notation*}
  We will have two universe sizes - one small and one large.
  We denote the category of small sets as $\set$ and the large sets as $\Set$.
  For example, we could take the small sets $\set$ to be those in $\Set$ bounded in cardinality
  by some inaccessible cardinal.
\end{notation*}

\subsection{Types}

% Assume an inaccessible cardinal $\lambda$. Write $\Set$ for the category of all sets. Say that a set $A$ is $\lambda$-small if $|A| < \lambda$.  Write $\Set_\lambda$ for the full
% subcategory of $\Set$ spanned by  $\lambda$-small sets.

Let $\catC$ be a small category, i.e.~a category whose class of objects is a $\set$ and
whose hom-classes are from $\set$.
% We do not assume that $\mathbb{C}$ is $\lambda$-small for
% the moment.
We write $\pshC$ for the category of presheaves over $\catC$,
\[
\pshC \defeq [\catC^\op, \Set]
\]

\medskip

% Because of the assumption of the existence of $\lambda$, $\pshC$ has additional structure. Let
% \[
% \Term \to \Type
% \]
% be the Hofmann-Streicher universe in $\pshC$ associated to $\lambda$. In particular,
% \[
% \Type(c) \defeq \{  A \co (\catC_{/c})^\op \to \Set_{\lambda} \}
% \]

\begin{defn}
  Following Awodey \cite{awodey2017naturalmodelshomotopytype},
  we say that a map $\tp : \Term \to \Type$ is presentable when
  any fiber of a representable is representable.
  In other words, given any $\Ga \in \catC$ and a map $A : \yo (\Ga) \to \Type$,
  there is some representable $\Ga \cdot A \in \catC$ and maps $\disp{A} : \Ga \cdot A \to \Ga$
  and $\var_{A} : \yo (\Ga \cdot A) \to \Term$ forming a pullback
  % https://q.uiver.app/#q=WzAsNCxbMCwwLCJcXHlvIChcXEdhIFxcY2RvdCBBKSJdLFswLDEsIlxceW8gKFxcR2EpIl0sWzEsMSwiXFxUeXBlIl0sWzEsMCwiXFxUZXJtIl0sWzAsMSwiXFx5byAoXFxkaXNwe0F9KSIsMl0sWzEsMiwiQSIsMl0sWzMsMl0sWzAsMywiXFx2YXJfQSJdXQ==
\[\begin{tikzcd}
	{\yo (\Ga \cdot A)} & \Term \\
	{\yo (\Ga)} & \Type
	\arrow["{\var_A}", from=1-1, to=1-2]
	\arrow["{\yo (\disp{A})}"', from=1-1, to=2-1]
	\arrow["\tp", from=1-2, to=2-2]
	\arrow["A"', from=2-1, to=2-2]
\end{tikzcd}\]
\end{defn}

\medskip

The Natural Model associated to a presentable map $\tp \co \Term \to \Type$ consists of
\begin{itemize}
\item contexts as objects $\Gamma, \Delta, \ldots \in \catC$,
\item a type in context $\yo (\Gamma)$ as a map $A \co \yo(\Gamma) \to \Type$,
\item a term of type $A$ in context $\Gamma$ as a map $a \co \yo(\Gamma) \to \Term$ such that
\[
\xymatrix{
 & \Term \ar[d]^{\tp} \\
\Gamma \ar[r]_-A \ar[ur]^{a} & \Type}
\]
commutes,
\item an operation called ``context extension'' which given a context $\Gamma$ and a type $A \co \yo(\Gamma) \to \Type$ produces a context $\Gamma\cdot A$ which fits into a pullback diagram below.
\[
\xymatrix{
\yo(\Gamma.A) \ar[r] \ar[d] & \Term \ar[d] \\
\yo(\Gamma) \ar[r]_-{A} & \Type}
\]
\end{itemize}

% In the internal type theory of $\pshC$, we write
% \[
%  (\Gamma) \ A \co \Type \qquad  (\Gamma) \ a \co A
% \]
% to mean that $A$ is a type in context $\Gamma$ and that $a$ is an element of type $A$ in context
% $\Gamma$, respectively.




{\bf Remark.}
Sometimes, we first construct a presheaf $X$ over $\Gamma$ and observe that it can be classified by a map into $\Type$. We write
\[
\xymatrix@C=1cm{
X \ar[r] \ar[d]& \Term \ar[d] \\
\yo(\Gamma) \ar[r]_-{\ulcorner X \urcorner} & \Type}
\]
to express this situation, i.e.~$X \cong \yo(\Gamma \cdot \ulcorner X \urcorner)$.

\medskip

\subsection{Pi types}

We will use $\Poly{\tp}$ to denote the polynomial endofunctor
(\cref{polynomial_endofunctor}) associated with our presentable map $\tp$.
Then an interpretation of $\Pi$ types consists of a pullback square
% https://q.uiver.app/#q=WzAsNCxbMCwwLCJcXFBvbHl7XFx0cH17XFxUZXJtfSJdLFswLDEsIlxcUG9seXtcXHRwfXtcXFR5cGV9Il0sWzEsMCwiXFxUZXJtIl0sWzEsMSwiXFxUeXBlIl0sWzAsMSwiXFxQb2x5e1xcdHB9e1xcdHB9IiwyXSxbMCwyLCJcXGxhIl0sWzIsMywiXFx0cCJdLFsxLDMsIlxcUGkiLDJdLFswLDMsIiIsMSx7InN0eWxlIjp7Im5hbWUiOiJjb3JuZXIifX1dXQ==
\[\begin{tikzcd}
	{\Poly{\tp}{\Term}} & \Term \\
	{\Poly{\tp}{\Type}} & \Type
	\arrow["\la", from=1-1, to=1-2]
	\arrow["{\Poly{\tp}{\tp}}"', from=1-1, to=2-1]
	\arrow["\lrcorner"{anchor=center, pos=0.125}, draw=none, from=1-1, to=2-2]
	\arrow["\tp", from=1-2, to=2-2]
	\arrow["\Pi"', from=2-1, to=2-2]
\end{tikzcd}\]

% TODO: details, interpretation of syntax

\subsection{Sigma types}

An interpretation of $\Si$ types consists of a pullback square
% https://q.uiver.app/#q=WzAsNCxbMCwwLCJRIl0sWzAsMSwiXFxQb2x5e1xcdHB9e1xcVHlwZX0iXSxbMSwwLCJcXFRlcm0iXSxbMSwxLCJcXFR5cGUiXSxbMCwxLCJcXHRwIFxcdHJpYW5nbGVsZWZ0IFxcdHAiLDJdLFswLDIsIlxccGFpciJdLFsyLDMsIlxcdHAiXSxbMSwzLCJcXFNpIiwyXSxbMCwzLCIiLDEseyJzdHlsZSI6eyJuYW1lIjoiY29ybmVyIn19XV0=
\[\begin{tikzcd}
	Q & \Term \\
	{\Poly{\tp}{\Type}} & \Type
	\arrow["\pair", from=1-1, to=1-2]
	\arrow["{\tp \triangleleft \tp}"', from=1-1, to=2-1]
	\arrow["\lrcorner"{anchor=center, pos=0.125}, draw=none, from=1-1, to=2-2]
	\arrow["\tp", from=1-2, to=2-2]
	\arrow["\Si"', from=2-1, to=2-2]
\end{tikzcd}\]

% TODO: details, interpretation of syntax

\subsection{Identity types}

To interpret the formation and introduction rules for identity types
we require a commutative square (this need not be pullback)
% https://q.uiver.app/#q=WzAsNCxbMCwwLCJcXFRlcm0iXSxbMCwxLCJcXHRwIFxcdGltZXNfXFxUeXBlIFxcdHAiXSxbMSwxLCJcXFR5cGUiXSxbMSwwLCJcXFRlcm0iXSxbMCwxLCJcXGRlIiwyXSxbMSwyLCJcXElkIiwyXSxbMywyLCJcXHRwIl0sWzAsMywiXFxyZWZsIl1d
\[\begin{tikzcd}
	\Term & \Term \\
	{\tp \times_\Type \tp} & \Type
	\arrow["\refl", from=1-1, to=1-2]
	\arrow["\de"', from=1-1, to=2-1]
	\arrow["\tp", from=1-2, to=2-2]
	\arrow["\Id"', from=2-1, to=2-2]
\end{tikzcd}\]
where $\de$ is the diagonal:
% https://q.uiver.app/#q=WzAsNSxbMCwwLCJcXFRlcm0iXSxbMSwxLCJcXHRwIFxcdGltZXNfXFxUeXBlIFxcdHAiXSxbMiwxLCJcXFRlcm0iXSxbMiwyLCJcXFR5cGUiXSxbMSwyLCJcXFRlcm0iXSxbMCwxLCJcXGRlIiwyXSxbMCwyXSxbMSwyXSxbMiwzLCJcXHRwIl0sWzQsMywiXFx0cCIsMl0sWzEsNF0sWzEsMywiIiwyLHsic3R5bGUiOnsibmFtZSI6ImNvcm5lciJ9fV0sWzAsNCwiIiwwLHsibGV2ZWwiOjIsInN0eWxlIjp7ImhlYWQiOnsibmFtZSI6Im5vbmUifX19XV0=
\[\begin{tikzcd}
	\Term \\
	& {\tp \times_\Type \tp} & \Term \\
	& \Term & \Type
	\arrow["\de"', from=1-1, to=2-2]
	\arrow[Rightarrow, no head, bend left, from=1-1, to=2-3]
	\arrow[Rightarrow, no head, bend right, from=1-1, to=3-2]
	\arrow[from=2-2, to=2-3]
	\arrow[from=2-2, to=3-2]
	\arrow["\lrcorner"{anchor=center, pos=0.125}, draw=none, from=2-2, to=3-3]
	\arrow["\tp", from=2-3, to=3-3]
	\arrow["\tp"', from=3-2, to=3-3]
\end{tikzcd}\]

Then let $I$ be the pullback.
We get a comparison map $\rho$
% https://q.uiver.app/#q=WzAsNSxbMCwwLCJcXFRlcm0iXSxbMSwyLCJcXHRwIFxcdGltZXNfXFxUeXBlIFxcdHAiXSxbMiwyLCJcXFR5cGUiXSxbMiwxLCJcXFRlcm0iXSxbMSwxLCJJIl0sWzAsMSwiXFxkZSIsMix7ImN1cnZlIjoyfV0sWzEsMiwiXFxJZCIsMl0sWzMsMiwiXFx0cCJdLFswLDMsIlxccmVmbCIsMCx7ImN1cnZlIjotM31dLFs0LDFdLFs0LDNdLFs0LDIsIiIsMix7InN0eWxlIjp7Im5hbWUiOiJjb3JuZXIifX1dLFswLDQsIlxccmhvIiwxLHsic3R5bGUiOnsiYm9keSI6eyJuYW1lIjoiZGFzaGVkIn19fV1d
\[\begin{tikzcd}
	\Term \\
	& I & \Term \\
	& {\tp \times_\Type \tp} & \Type
	\arrow["\rho"{description}, dashed, from=1-1, to=2-2]
	\arrow["\refl", bend left, from=1-1, to=2-3]
	\arrow["\de", bend right, from=1-1, to=3-2]
	\arrow[from=2-2, to=2-3]
	\arrow[from=2-2, to=3-2]
	\arrow["\lrcorner"{anchor=center, pos=0.125}, draw=none, from=2-2, to=3-3]
	\arrow["\tp", from=2-3, to=3-3]
	\arrow["\Id"', from=3-2, to=3-3]
\end{tikzcd}\]

Then view $\rho : \tp \to q$ as a map in the slice over $\Type$.
% https://q.uiver.app/#q=WzAsNSxbMCwwLCJcXFRlcm0iXSxbMSwyLCJcXHRwIFxcdGltZXNfXFxUeXBlIFxcdHAiXSxbMSwxLCJJIl0sWzEsMywiXFxUZXJtIl0sWzEsNCwiXFxUeXBlIl0sWzAsMSwiXFxkZSIsMSx7ImN1cnZlIjoxfV0sWzIsMV0sWzAsMiwiXFxyaG8iLDEseyJzdHlsZSI6eyJib2R5Ijp7Im5hbWUiOiJkYXNoZWQifX19XSxbMSwzLCJcXGZzdCIsMl0sWzMsNF0sWzAsNCwiXFx0cCIsMix7ImN1cnZlIjozfV0sWzAsMywiIiwyLHsiY3VydmUiOjIsImxldmVsIjoyLCJzdHlsZSI6eyJoZWFkIjp7Im5hbWUiOiJub25lIn19fV0sWzIsNCwicSIsMCx7ImN1cnZlIjotNX1dXQ==
\[\begin{tikzcd}
	\Term \\
	& I \\
	& {\tp \times_\Type \tp} \\
	& \Term \\
	& \Type
	\arrow["\rho"{description}, dashed, from=1-1, to=2-2]
	\arrow["\de"{description}, bend right = 10, from=1-1, to=3-2]
	\arrow[bend right = 25, Rightarrow, no head, from=1-1, to=4-2]
	\arrow["\tp"', bend right = 40, from=1-1, to=5-2]
	\arrow[from=2-2, to=3-2]
	\arrow["q", bend left = 50, from=2-2, to=5-2]
	\arrow["\fst"', from=3-2, to=4-2]
	\arrow[from=4-2, to=5-2]
\end{tikzcd}\]

Now (by \cref{polynomial_on_slice_map})
applying $\Poly{-} : (\pshC / \Type) ^{\op} \to [\pshC, \pshC]$ to $\rho : \tp \to q$
gives us a naturality square (this also need not be pullback).
% https://q.uiver.app/#q=WzAsNCxbMCwwLCJcXFBvbHl7cX0ge1xcVGVybX0iXSxbMCwxLCJcXFBvbHl7cX0ge1xcVHlwZX0iXSxbMSwxLCJcXFBvbHl7XFx0cH0ge1xcVGVybX0iXSxbMSwwLCJcXFBvbHl7XFx0cH0ge1xcVGVybX0iXSxbMCwxLCJcXFBvbHl7cX0ge1xcdHB9IiwyXSxbMSwyLCJcXFN0YXJ7XFxyaG99X3tcXFR5cGV9IiwyXSxbMywyLCJcXFBvbHl7XFx0cH0ge1xcdHB9Il0sWzAsMywiXFxTdGFye1xccmhvfV97XFxUZXJtfSJdXQ==
\[\begin{tikzcd}
	{\Poly{q} {\Term}} & {\Poly{\tp} {\Term}} \\
	{\Poly{q} {\Type}} & {\Poly{\tp} {\Term}}
	\arrow["{\Star{\rho}_{\Term}}", from=1-1, to=1-2]
	\arrow["{\Poly{q} {\tp}}"', from=1-1, to=2-1]
	\arrow["{\Poly{\tp} {\tp}}", from=1-2, to=2-2]
	\arrow["{\Star{\rho}_{\Type}}"', from=2-1, to=2-2]
\end{tikzcd}\]
Taking the pullback $T$ and the comparison map $\ep$ we have
% https://q.uiver.app/#q=WzAsNSxbMCwwLCJcXFBvbHl7cX0ge1xcVGVybX0iXSxbMSwyLCJcXFBvbHl7cX0ge1xcVHlwZX0iXSxbMiwyLCJcXFBvbHl7XFx0cH0ge1xcVGVybX0iXSxbMiwxLCJcXFBvbHl7XFx0cH0ge1xcVGVybX0iXSxbMSwxLCJUIl0sWzAsMSwiXFxQb2x5e3F9IHtcXHRwfSIsMix7ImN1cnZlIjoyfV0sWzEsMiwiXFxTdGFye1xccmhvfV97XFxUeXBlfSIsMl0sWzMsMiwiXFxQb2x5e1xcdHB9IHtcXHRwfSJdLFswLDMsIlxcU3RhcntcXHJob31fe1xcVGVybX0iLDAseyJjdXJ2ZSI6LTN9XSxbNCwxXSxbNCwzXSxbNCwyLCIiLDIseyJzdHlsZSI6eyJuYW1lIjoiY29ybmVyIn19XSxbMCw0LCJcXGVwIiwxLHsic3R5bGUiOnsiYm9keSI6eyJuYW1lIjoiZGFzaGVkIn19fV1d
\[\begin{tikzcd}
	{\Poly{q} {\Term}} \\
	& T & {\Poly{\tp} {\Term}} \\
	& {\Poly{q} {\Type}} & {\Poly{\tp} {\Term}}
	\arrow["\ep"{description}, dashed, from=1-1, to=2-2]
	\arrow["{\Star{\rho}_{\Term}}", bend left, from=1-1, to=2-3]
	\arrow["{\Poly{q} {\tp}}"', bend right, from=1-1, to=3-2]
	\arrow[from=2-2, to=2-3]
	\arrow[from=2-2, to=3-2]
	\arrow["\lrcorner"{anchor=center, pos=0.125}, draw=none, from=2-2, to=3-3]
	\arrow["{\Poly{\tp} {\tp}}", from=2-3, to=3-3]
	\arrow["{\Star{\rho}_{\Type}}"', from=3-2, to=3-3]
\end{tikzcd}\]
Finally, we require a section $J : T \to \Poly{q}{\Term}$ of $\ep$,
to interpret the identity elimination rule.

\subsection{A type of small types}

We now wish to formulate a condition that allows us to have a type of small types, written $\U$, not just {\em judgement} expressing that something is a type. With this notation, the judgements that we would like to derive is
\[
 \U \co \Type \qquad
 \begin{prooftree}
 a \co \U
 \justifies
 \El(a) \co \Type
 \end{prooftree}
\]

(A sufficient and natural condition for this seems to be that we now have another inaccessible cardinal $\kappa$, with $\kappa < \lambda$.)

In the Natural Model, a universe $\U$ is postulated by a map
\[
\pi \co \E \to \U
\]

In the Natural Model:
\begin{itemize}
\item There is a pullback diagram of the form
\begin{equation}
\xymatrix{
\U \ar[r] \ar[d] & \Term \ar[d] \\
1 \ar[r]_-{\ulcorner \U \urcorner } & \Type }
\end{equation}
\item There is an inclusion of $\U$ into $\Type$
\[
\El \co \U \rightarrowtail \Type
\]
\item $\pi : \E \to \U$ is obtained as pullback of $\tp$; There is a pullback diagram
\begin{equation}\label{diag:universe-pullback}
\xymatrix{
E \ar@{>->}[r] \ar[d] & \Term \ar[d] \\
\U \ar@{>->}[r]_-{\El} & \Type }
\end{equation}
 \end{itemize}

With the notation above, we get
\[
\xymatrix{
\yo (\Gamma.\El(a)) \ar[r] \ar[d] & \E \ar[r] \ar[d] & \Term \ar[d] \\
\yo (\Gamma) \ar[r]_a  \ar@/_2pc/[rr]_-{A} & \U \ar[r]_{\El} & \Type}
\]
Both squares above are pullback squares.

\subsection{Defining the dependent function and product types for the universe}

Take the pullback diagram \cref{diag:universe-pullback}. That is a morphism in the category of polynomials. We have a cartesian natural transformation $P_\pi \to P_{\tp}$ induced by the pullback \cref{diag:universe-pullback}. This cartesian natural transformation induces the diagrams of the left cube in below; all of the squares in the left cube are pullback squares.

Now, consider the right cube only. The right-side face of the cube is a pullback square due to the definition of $\pi$. The left-side is also because polynomials are left exact functors and therefore they preserve pullbacks. Observe also that the front face is a pullback square by the definition of $\Pi$ for $\Type$.

\[
\begin{tikzcd}[row sep=scriptsize, column sep=scriptsize]
& P_\pi E \arrow[dd] \arrow[rr] \arrow[ld] & & P_\tp E \arrow[dl]  \arrow[rr] \arrow[dd] & & E \arrow[dl] \arrow[dd] \\
P_\pi \Term \arrow[rr, crossing over] \arrow[dd] && P_\tp \Term \arrow[rr, crossing over, pos=0.6, "\lambda"] \arrow[dd, crossing over] & & \Term \\
& P_\pi \U \arrow[rr] \arrow[ld] &&  P_\tp U \arrow[dl] \arrow[rr, dashed] & & U \arrow[dl] \\
P_\pi \Type \arrow[rr] && P_\tp \Type \arrow[rr, swap, "\Pi"] & & \Type \arrow[from=uu, crossing over]\\
\end{tikzcd}
\]

We define $\Pi_ U \colon P_\pi U \to U$ as a dashed arrow which makes the bottom square in the right cube commute, that is the following square commutes.

\begin{equation}\label{diag:universe-pullback-3}
\xymatrix{
P_\tp \U \ar@{->}[r] \ar@{-->}[d]_{\Pi_\U} & P_\tp \Type \ar[d]^{\Pi_\Type} \\
\U \ar@{->}[r]_-{\El} & \Type }
\end{equation}

By the universal property of the pullback which defines $\pi$ we get a unique arrow $ P_\tp E \to E$ which we shall name $\lambda_\U$. By the pullback pasting lemma it follows that the square involving $\Pi_ U$ and $\lambda_U$ is a pullback square.

\begin{equation}\label{diag:universe-pullback-3}
\xymatrix{
P_\tp \E \ar@{->}[r]^{\lambda_\U} \ar[d]_{P_\tp \pi} & E \ar[d]^{P_\tp \pi} \\
P_\tp \U \ar@{->}[r]_-{\Pi_\U} & \U }
\end{equation}

This concludes the construction of $\Pi$ type former for the universe $\U$. The only data we needed to supply for the definition of $\Pi_U$ was a lift of $\Pi \colon P_\tp \Type \to \Type$ to $U$.


% \subsection{The Universe in Embedded Type Theory (HoTT0) and the relationship to the Natural Model}
