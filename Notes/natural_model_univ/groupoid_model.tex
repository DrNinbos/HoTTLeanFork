
In this section we construct a natural model in $\Pshgrpd$ the presheaf category
indexed by the category $\grpd$ of (small) groupoids.
We will build the classifier for display maps in the style of
Hofmann and Streicher \cite{hofmannstreicher1996} and Awodey \cite{awodey2023hofmannstreicheruniverses}.
To interpret the type constructors,
we will make use of the weak factorization system on $\grpd$ -
which comes from restricting the ``classical Quillen model structure'' on
$\cat$ \cite{joyalnlabmodelstructuresoncat} to $\grpd$.

\medskip

\subsection{Classifying display maps}

\medskip

\begin{notation*}
  We will have two universe sizes - one small and one large.
  We denote the category of small sets as $\set$ and the large sets as $\Set$
  (in the previous sections this would have been $\Set_{\lambda}$ and $\Set$ respectively).
  We denote the category of small categories as $\cat$ and the large categories as $\Cat$.
  We denote the category of small groupoids as $\grpd$. %% and the large groupoids as $\GRPD$.

  We are primarily working in the category of large presheaves indexed by small groupoids,
  which we will denote by
  \[ \Pshgrpd = [\grpd^{\op}, \Set]\]

  In this section, $\Term$ and $\Type$ and so on will refer to the natural model semantics in this
  specific model.
\end{notation*}

\medskip

\begin{defn}[Pointed]
  We will take the category of pointed small categories $\ptcat$
  to have objects as pairs $(\catC \in \cat, c \in \catC)$
  and morphisms as pairs
  \[ (F : \catC_{1} \to \catC_{0}, \phi : F c_{1} \to c_{0}) \co (\catC_{1}, c_{1}) \to (\catC_{0}, c_{0})\]
  Then the category of pointed small groupoids $\ptgrpd$ will be the full subcategory
  of objects $(\Ga, c)$ with $\Ga$ a groupoid.
\end{defn}

\medskip

\begin{defn}[The display map classifier]
  We would like to define a natural transformation in
  $\Pshgrpd$
  \[ \tp \co \Term \to \Type \]
  with representable fibers.

  Consider the functor that forgets the point
  \[
    U \co \ptgrpd \to \grpd
    \quad \quad
    \text{
      in \quad $\Cat$.
    }
  \]
  If we apply the Yoneda embedding $\yo \co \Cat \to \PshCat$ to $U$
  we obtain
  \[ U \circ \co [ - , \ptgrpd] \to [ - , \grpd ]
    \quad \quad
    \text{
      in \quad $\PshCat$.
    }
  \]
  Since any small groupoid is also a large category $\grpd \hookrightarrow \Cat$,
  we can restrict $\Cat$ indexed presheaves to be $\grpd$ indexed presheaves.
  We define $\tp \co \Term \to \Type$ as the image of $U \circ$ under this restriction.
  % https://q.uiver.app/#q=WzAsNixbMCwwLCJcXENhdCJdLFsxLDAsIlxcUHNoQ2F0Il0sWzIsMCwiXFxQc2hncnBkIl0sWzAsMSwiXFxncnBkIl0sWzEsMSwiWy0sXFxncnBkXSJdLFsyLDEsIlxcVHlwZSJdLFswLDEsInkiXSxbMSwyLCJcXHRleHRzZntyZXN9Il0sWzMsNCwiIiwwLHsic3R5bGUiOnsidGFpbCI6eyJuYW1lIjoibWFwcyB0byJ9fX1dLFs0LDUsIiIsMCx7InN0eWxlIjp7InRhaWwiOnsibmFtZSI6Im1hcHMgdG8ifX19XV0=
  \[\begin{tikzcd}[row sep = tiny]
    \Cat & \PshCat & \Pshgrpd \\
    \grpd & {[-,\grpd]} & \Type
    \arrow["\yo", from=1-1, to=1-2]
    \arrow["{\textsf{res}}", from=1-2, to=1-3]
    \arrow[maps to, from=2-1, to=2-2]
    \arrow[maps to, from=2-2, to=2-3]
  \end{tikzcd}\]
  Note that $\Term$ and $\Type$ are not representable in $\Pshgrpd$.
\end{defn}

\medskip

\begin{rmk}
  By Yoneda we can identify maps with representable domain into the
  type classifier
  \[ A : \yo \Ga \to \Type \quad \quad \text{ in } \quad \Pshgrpd \]
  with functors
  \[ A : \Ga \to \grpd \quad \quad \text{ in } \quad \Cat \]

\end{rmk}

\medskip

\begin{defn}[Grothendieck construction]
  From $\catC$ a small category and $F : \catC \to \cat$
  a functor, we construct a small category $\int F$.
  For any $c$ in $\catC$ we refer to $F c$ as the fiber over $c$.
  The objects of $\int F$ consist of pairs $(c \in \catC, x \in F c)$,
  and morphisms between $(c, x)$ and $(d, y)$
  are pairs $(f : c \to d, \phi : F \, f \, x \to y)$.
  This makes the following pullback in $\Cat$
  % https://q.uiver.app/#q=WzAsMTEsWzEsMSwiXFxpbnQgRiJdLFsxLDIsIlxcY2F0QyJdLFsyLDEsIlxccHRjYXQiXSxbMiwyLCJcXGNhdCJdLFswLDEsIihjLHgpIl0sWzAsMiwiYyJdLFsxLDAsIihjLHgpIl0sWzIsMCwiKEZjLCB4KSJdLFsxLDNdLFszLDEsIihDLGMpIl0sWzMsMiwiQyJdLFswLDFdLFswLDJdLFsyLDNdLFsxLDMsIkYiLDJdLFs0LDUsIiIsMix7InN0eWxlIjp7InRhaWwiOnsibmFtZSI6Im1hcHMgdG8ifX19XSxbNiw3LCIiLDIseyJzdHlsZSI6eyJ0YWlsIjp7Im5hbWUiOiJtYXBzIHRvIn19fV0sWzAsMywiIiwyLHsic3R5bGUiOnsibmFtZSI6ImNvcm5lciJ9fV0sWzksMTAsIiIsMix7InN0eWxlIjp7InRhaWwiOnsibmFtZSI6Im1hcHMgdG8ifX19XV0=
  \[\begin{tikzcd}
    & {(c,x)} & {(Fc, x)} \\
    {(c,x)} & {\int F} & \ptcat & {(C,c)} \\
    c & \catC & \cat & C \\
    & {}
    \arrow[maps to, from=1-2, to=1-3]
    \arrow[maps to, from=2-1, to=3-1]
    \arrow[from=2-2, to=2-3]
    \arrow[from=2-2, to=3-2]
    \arrow["\lrcorner"{anchor=center, pos=0.125}, draw=none, from=2-2, to=3-3]
    \arrow[from=2-3, to=3-3]
    \arrow[maps to, from=2-4, to=3-4]
    \arrow["F"', from=3-2, to=3-3]
  \end{tikzcd}\]
\end{defn}

\medskip

\begin{defn}[Grothendieck construction for groupoids]
  Let $\Ga$ be a groupoid and $A \co \Ga \to \grpd$ a functor,
  we can compose $F$ with the inclusion $i \co \grpd \hookrightarrow \Cat$
  and form the Grothendieck construction which we denote as
  \[ \Ga \cdot A := \int i \circ A \quad \quad \disp{A} \co \Ga \cdot A \to \Ga\]
  This is also a small groupoid since the underlying morphisms are
  pairs of morphisms from groupoids $\Ga$ and $A x$ for $x \in \Ga$.
  Furthermore the pullback factors through (pointed) groupoids.
  % https://q.uiver.app/#q=WzAsNyxbMCwwLCJcXEdhIFxcY2RvdCBBIl0sWzAsMSwiXFxHYSJdLFsxLDAsIlxccHRncnBkIl0sWzEsMSwiXFxncnBkIl0sWzAsMl0sWzIsMCwiXFxwdENhdCJdLFsyLDEsIlxcQ2F0Il0sWzAsMV0sWzAsMl0sWzIsM10sWzEsMywiQSIsMl0sWzAsMywiIiwyLHsic3R5bGUiOnsibmFtZSI6ImNvcm5lciJ9fV0sWzIsNV0sWzMsNl0sWzUsNl0sWzIsNiwiIiwxLHsic3R5bGUiOnsibmFtZSI6ImNvcm5lciJ9fV1d
  \[\begin{tikzcd}
    {\Ga \cdot A} & \ptgrpd & \ptcat \\
    \Ga & \grpd & \cat \\
    {}
    \arrow[from=1-1, to=1-2]
    \arrow["\disp{A}"', from=1-1, to=2-1]
    \arrow["\lrcorner"{anchor=center, pos=0.125}, draw=none, from=1-1, to=2-2]
    \arrow[from=1-2, to=1-3]
    \arrow[from=1-2, to=2-2]
    \arrow["\lrcorner"{anchor=center, pos=0.125}, draw=none, from=1-2, to=2-3]
    \arrow[from=1-3, to=2-3]
    \arrow["A"', from=2-1, to=2-2]
    \arrow[from=2-2, to=2-3]
  \end{tikzcd}\]
\end{defn}

\medskip

\begin{cor}[The display map classifier is presentable]
  For any small groupoid $\Ga$ and $A : \yo \Ga \to \Type$,
  the pullback of $\tp$ along $A$ can be given by the representable
  map $\yo \disp{A}$.
  % https://q.uiver.app/#q=WzAsNSxbMCwwLCJcXHlvIFxcR2EgXFxjZG90IEEiXSxbMCwxLCJcXHlvIFxcR2EiXSxbMSwwLCJcXFRlcm0iXSxbMSwxLCJcXFR5cGUiXSxbMCwyXSxbMCwxLCJcXGRpc3AgQSIsMl0sWzAsMl0sWzIsMywiXFx0cCIsMl0sWzEsMywiQSIsMl0sWzAsMywiIiwyLHsic3R5bGUiOnsibmFtZSI6ImNvcm5lciJ9fV1d
  \[\begin{tikzcd}
    {\yo \Ga \cdot A} & \Term \\
    {\yo \Ga} & \Type \\
    {}
    \arrow[from=1-1, to=1-2]
    \arrow["{\yo \disp A}"', from=1-1, to=2-1]
    \arrow["\lrcorner"{anchor=center, pos=0.125}, draw=none, from=1-1, to=2-2]
    \arrow["\tp"', from=1-2, to=2-2]
    \arrow["A"', from=2-1, to=2-2]
  \end{tikzcd}\]
\end{cor}
\begin{proof}
  Consider the pullback in $\Cat$
  % https://q.uiver.app/#q=WzAsNSxbMCwwLCJcXEdhIFxcY2RvdCBBIl0sWzAsMSwiXFxHYSJdLFsxLDAsIlxccHRncnBkIl0sWzEsMSwiXFxncnBkIl0sWzAsMl0sWzAsMV0sWzAsMl0sWzIsM10sWzEsMywiQSIsMl0sWzAsMywiIiwyLHsic3R5bGUiOnsibmFtZSI6ImNvcm5lciJ9fV1d
  \[\begin{tikzcd}
    {\Ga \cdot A} & \ptgrpd \\
    \Ga & \grpd \\
    {}
    \arrow[from=1-1, to=1-2]
    \arrow[from=1-1, to=2-1]
    \arrow["\lrcorner"{anchor=center, pos=0.125}, draw=none, from=1-1, to=2-2]
    \arrow[from=1-2, to=2-2]
    \arrow["A"', from=2-1, to=2-2]
  \end{tikzcd}\]
  We send this square along $\textsf{res} \circ \yo$ in the following
  % https://q.uiver.app/#q=WzAsNSxbMiwxLCJcXFBzaGdycGQiXSxbMSwyXSxbMiwwLCJcXFBzaENhdCJdLFswLDAsIlxcQ2F0Il0sWzAsMSwiXFxncnBkIl0sWzMsMiwiXFx5byJdLFsyLDAsIlxcdGV4dHNme3Jlc30iXSxbMywwXSxbNCwzXSxbNCwwLCJcXHlvIl1d
  \[\begin{tikzcd}
    \Cat && \PshCat \\
    \grpd && \Pshgrpd \\
    & {}
    \arrow["\yo", from=1-1, to=1-3]
    \arrow[from=1-1, to=2-3]
    \arrow["{\textsf{res}}", from=1-3, to=2-3]
    \arrow[from=2-1, to=1-1]
    \arrow["\yo", from=2-1, to=2-3]
  \end{tikzcd}\]
  The Yoneda embedding $\yo : \Cat \to \PshCat$ preserves pullbacks,
  as does $\textsf{res}$ since it is a right adjoint
  (with left Kan extension $\io_{!} \dashv \textsf{res}_{\io}$).
\end{proof}

\medskip

\subsection{Groupoid fibrations}

\begin{defn}[Fibration]
  Let $p : \catC_{1} \to \catC_{0}$ be a functor.
  We say $p$ is a \textit{split Grothendieck fibration}
  if we have a dependent function $\lift{a}{f}$ satisfying the following:
  for any object $a$ in $\catC_{1}$ and morphism $f : p \, a \to y$
  in the base $\catC_{0}$
  we have $\lift{a}{f} : a \to b$ in $\catC_{1}$ such that $p (\lift{a}{f}) = f$
  and moreover $\lift{a}{g \circ f} = \lift{b}{g} \circ \lift{a}{f}$
  % https://q.uiver.app/#q=WzAsNCxbMCwwLCJhIl0sWzEsMCwiYiJdLFswLDEsIngiXSxbMSwxLCJ5Il0sWzAsMSwiXFxsaWZ0e2F9e2Z9IiwwLHsic3R5bGUiOnsiYm9keSI6eyJuYW1lIjoiZGFzaGVkIn19fV0sWzAsMiwiIiwyLHsic3R5bGUiOnsidGFpbCI6eyJuYW1lIjoibWFwcyB0byJ9fX1dLFsyLDMsImYiLDJdLFsxLDMsIiIsMCx7InN0eWxlIjp7InRhaWwiOnsibmFtZSI6Im1hcHMgdG8ifSwiYm9keSI6eyJuYW1lIjoiZGFzaGVkIn19fV0sWzQsNiwiIiwwLHsic2hvcnRlbiI6eyJzb3VyY2UiOjIwLCJ0YXJnZXQiOjIwfSwic3R5bGUiOnsidGFpbCI6eyJuYW1lIjoibWFwcyB0byJ9LCJib2R5Ijp7Im5hbWUiOiJkYXNoZWQifX19XV0=
  \[\begin{tikzcd}
    a & b \\
    x & y
    \arrow[""{name=0, anchor=center, inner sep=0}, "{\lift{a}{f}}", dashed, from=1-1, to=1-2]
    \arrow[maps to, from=1-1, to=2-1]
    \arrow[dashed, maps to, from=1-2, to=2-2]
    \arrow[""{name=1, anchor=center, inner sep=0}, "f"', from=2-1, to=2-2]
    \arrow[shorten <=4pt, shorten >=4pt, Rightarrow, dashed, maps to, from=0, to=1]
  \end{tikzcd}\]

  In particular, we are intereseted in split Grothendieck fibrations of groupoids,
  which are the same as \textit{isofibrations}
  (replace all the morphisms with isomorphisms in the definition).

  Unless specified otherwise,
  by a \textit{fibration} we will mean a split Grothendieck fibration of groupoids.
  Let us denote the category of fibrations over a groupoid $\Ga$ as $\Fib_{\Ga}$,
  which is a full subcategory of the slice $\grpd / \Ga$.
  We will decorate an arrow with $\twoheadrightarrow$
  to indicate it is a fibration.
\end{defn}

\medskip

Note that $\disp{A} \co \Ga \cdot A \to \Ga$ is a fibration,
since for any $(x \in \Ga, a \in A \, x)$ and $f \co x \to y$ in $\Ga$
we have a morphism $(f, \id_{A \, f \, a}) : (x, a) \to (y, A \, f \, a)$
lifting $f$. Furthermore

\begin{prop}
  There is an adjoint equivalence
 % https://q.uiver.app/#q=WzAsMixbMCwwLCJbXFxHYSwgXFxncnBkXSJdLFsyLDAsIlxcRmliX1xcR2EiXSxbMCwxLCJcXGRpc3B7fSIsMCx7Im9mZnNldCI6LTJ9XSxbMSwwLCJcXHRleHRzZntmaWJlcn0iLDAseyJvZmZzZXQiOi0yfV0sWzAsMSwiXFxzaW1lcSIsMSx7InN0eWxlIjp7ImJvZHkiOnsibmFtZSI6Im5vbmUifSwiaGVhZCI6eyJuYW1lIjoibm9uZSJ9fX1dXQ==
  \[\begin{tikzcd}
    {[\Ga, \grpd]} && {\Fib_\Ga}
    \arrow["{\disp{}}", shift left=2, from=1-1, to=1-3]
    \arrow["\simeq"{description}, draw=none, from=1-1, to=1-3]
    \arrow["{\fiber}", shift left=2, from=1-3, to=1-1]
  \end{tikzcd}\]
  where for each fibration $\de : \De \to \Ga$ and each
  object $x \in \Ga$
  \[\fiber_{\de} \, x = \text{\emph {full subcategory}} \, \{a \in \De \st \de \, a = x\}\]
  It follows that all fibrations are pullbacks of the classifier
  $U : \ptgrpd \to \grpd$, when viewed as morphisms in $\Cat$.
  From now on, we will use $\disp{A}$ to represent any groupoid fibration,
  which we can adjust up to isomorphism using this equivalence.
\end{prop}

\medskip

\begin{prop}[Pullback of fibrations]
  Let $\si : \De \to Ga$ be a functor between groupoids.
  Since display maps are pullbacks
  of the classifier $U : \ptgrpd \to \grpd$ we have
  the pasting diagram
  % https://q.uiver.app/#q=WzAsNixbMSwwLCJcXEdhLkEiXSxbMSwxLCJcXEdhIl0sWzAsMSwiXFxEZSJdLFswLDAsIlxcRGUuQVxcc2kiXSxbMiwxLCJcXGdycGQiXSxbMiwwLCJcXHB0Z3JwZCJdLFswLDEsIlxcZGlzcHtBfSIsMSx7InN0eWxlIjp7ImhlYWQiOnsibmFtZSI6ImVwaSJ9fX1dLFsyLDEsIlxcc2kiLDJdLFsxLDQsIkEiLDJdLFswLDVdLFs1LDRdLFswLDQsIiIsMSx7InN0eWxlIjp7Im5hbWUiOiJjb3JuZXIifX1dLFszLDEsIiIsMSx7InN0eWxlIjp7Im5hbWUiOiJjb3JuZXIifX1dLFszLDIsIlxcZGlzcHtBXFxzaX0iLDEseyJzdHlsZSI6eyJoZWFkIjp7Im5hbWUiOiJlcGkifX19XSxbMywwLCIiLDEseyJzdHlsZSI6eyJib2R5Ijp7Im5hbWUiOiJkYXNoZWQifX19XSxbMyw1LCIiLDEseyJjdXJ2ZSI6LTJ9XV0=
  \[\begin{tikzcd}
    {\De.A\si} & {\Ga.A} & \ptgrpd \\
    \De & \Ga & \grpd
    \arrow[dashed, from=1-1, to=1-2]
    \arrow[bend left, from=1-1, to=1-3]
    \arrow["{\disp{A\si}}"{description}, two heads, from=1-1, to=2-1]
    \arrow["\lrcorner"{anchor=center, pos=0.125}, draw=none, from=1-1, to=2-2]
    \arrow[from=1-2, to=1-3]
    \arrow["{\disp{A}}"{description}, two heads, from=1-2, to=2-2]
    \arrow["\lrcorner"{anchor=center, pos=0.125}, draw=none, from=1-2, to=2-3]
    \arrow[from=1-3, to=2-3]
    \arrow["\si"', from=2-1, to=2-2]
    \arrow["A"', from=2-2, to=2-3]
  \end{tikzcd}\]

  Note that this avoids coherence issues, since we take the pullback to be
  the map $\disp{A \si}$ specifically.
  It follows that fibrations are stable under pullback along all groupoid functors.
  % Then define (strict) pullback of fibrations in the following way
  % https://q.uiver.app/#q=WzAsNCxbMCwwLCJbXFxHYSwgXFxncnBkXSJdLFsyLDAsIlxcRmliX1xcR2EiXSxbMCwxLCJbXFxEZSwgXFxncnBkXSJdLFsyLDEsIlxcRmliX1xcRGUiXSxbMSwwLCJcXHRleHRzZntmaWJlcn0iLDJdLFswLDIsIlxcY2lyYyBcXHNpIiwyXSxbMiwzLCJcXGRpc3B7fSIsMl0sWzEsMywiXFxzaV4qIiwwLHsic3R5bGUiOnsiYm9keSI6eyJuYW1lIjoiZGFzaGVkIn19fV1d
  \[\begin{tikzcd}
    {[\Ga, \grpd]} && {\Fib_\Ga} \\
    {[\De, \grpd]} && {\Fib_\De}
    \arrow["{\circ \si}"', from=1-1, to=2-1]
    \arrow["{\textsf{fiber}}"', from=1-3, to=1-1]
    \arrow["{\si^*}", dashed, from=1-3, to=2-3]
    \arrow["{\disp{}}"', from=2-1, to=2-3]
  \end{tikzcd}\]

\end{prop}

\medskip

\begin{defn}[Composition of fibrations]
  The composition of two fibrations is a fibration.
  % https://q.uiver.app/#q=WzAsMyxbMCwwLCJcXEdhIFxcY2RvdCBBIFxcY2RvdCBCIl0sWzAsMSwiXFxHYSBcXGNkb3QgQSJdLFsxLDEsIlxcR2EiXSxbMCwxLCIiLDAseyJzdHlsZSI6eyJoZWFkIjp7Im5hbWUiOiJlcGkifX19XSxbMSwyLCIiLDAseyJzdHlsZSI6eyJoZWFkIjp7Im5hbWUiOiJlcGkifX19XSxbMCwyLCIiLDIseyJzdHlsZSI6eyJib2R5Ijp7Im5hbWUiOiJkYXNoZWQifSwiaGVhZCI6eyJuYW1lIjoiZXBpIn19fV1d
  \[\begin{tikzcd}
    {\Ga \cdot A \cdot B} \\
    {\Ga \cdot A} & \Ga
    \arrow[two heads, from=1-1, to=2-1]
    \arrow[dashed, two heads, from=1-1, to=2-2]
    \arrow[two heads, from=2-1, to=2-2]
  \end{tikzcd}\]
  Then given $A : \Ga \to \grpd$ and $B : \Ga\cdot A \to \grpd$ we define
  \[ \Si_{A}B := \fiber (\disp{B} \circ \disp{A}) : \Ga \to \grpd\]

  Let us also define the natural transformation
  $\fst : \Si_{A}B \to A$ by
  % https://q.uiver.app/#q=WzAsNyxbMCwwLCJcXGRpc3B7QX1cXGNpcmMgXFxkaXNwe0J9Il0sWzAsMiwiXFxkaXNwe0F9Il0sWzEsMV0sWzIsMV0sWzMsMCwiXFxTaV9BIEIiXSxbMywyLCJcXGZpYmVyIChcXGRpc3B7QX0pIl0sWzUsMSwiQSJdLFswLDEsIlxcZGlzcHtCfSIsMV0sWzIsMywiXFxmaWJlciIsMCx7InN0eWxlIjp7InRhaWwiOnsibmFtZSI6Im1hcHMgdG8ifX19XSxbNCw1LCJcXGZpYmVyKFxcZGlzcCBCKSIsMV0sWzQsNiwiXFxmc3QiLDEseyJzdHlsZSI6eyJib2R5Ijp7Im5hbWUiOiJkYXNoZWQifX19XSxbNSw2LCJcXGNvbmciLDFdXQ==
\[\begin{tikzcd}
	{\disp{A}\circ \disp{B}} &&& {\Si_A B} \\
	& {} & {} &&& A \\
	{\disp{A}} &&& {\fiber (\disp{A})}
	\arrow["{\disp{B}}"{description}, from=1-1, to=3-1]
	\arrow["\fst"{description}, dashed, from=1-4, to=2-6]
	\arrow["{\fiber(\disp B)}"{description}, from=1-4, to=3-4]
	\arrow["\fiber", maps to, from=2-2, to=2-3]
	\arrow["\cong"{description}, from=3-4, to=2-6]
\end{tikzcd}\]
\end{defn}

\medskip

\medskip

\begin{defn}[Pushforward of fibrations]
  Given $A : \Ga \to \grpd$ and $B : \Ga \cdot A \to \grpd$
  we will define $\Pi_{A} B : \Ga \to \grpd$ such that
  for any $C : \Ga \to \grpd$ we have an isomorphism
  \[ [\Ga \cdot A, \grpd](\disp{A}\circ C, B) \iso
    [\Ga, \grpd](C, \Pi_{A} B)\]
  natural in both $B$ and $C$.
  Stated in terms of fibrations we have
  % https://q.uiver.app/#q=WzAsNCxbMCwwLCJcXEdhIFxcY2RvdCBBIFxcY2RvdCBCIl0sWzAsMSwiXFxHYSBcXGNkb3QgQSJdLFsxLDEsIlxcR2EiXSxbMSwwLCJcXEdhIFxcY2RvdCBcXFBpX0EgQiJdLFswLDEsIiIsMCx7InN0eWxlIjp7ImhlYWQiOnsibmFtZSI6ImVwaSJ9fX1dLFsxLDIsIiIsMCx7InN0eWxlIjp7ImhlYWQiOnsibmFtZSI6ImVwaSJ9fX1dLFszLDIsIiIsMix7InN0eWxlIjp7ImhlYWQiOnsibmFtZSI6ImVwaSJ9fX1dXQ==
  \[\begin{tikzcd}
    {\Ga \cdot A \cdot B} & {\Ga \cdot \Pi_A B} \\
    {\Ga \cdot A} & \Ga
    \arrow["\disp{B}"', two heads, from=1-1, to=2-1]
    \arrow[two heads, dashed, from=1-2, to=2-2]
    \arrow["\disp{A}"', two heads, from=2-1, to=2-2]
  \end{tikzcd}\]
  with the universal property of pushforward
  \[ \Fib_{\Ga \cdot A}(\disp{A}^{*}\disp{C}, \disp{B}) \iso \Fib_{\Ga}(\disp{C}, \disp{\Pi_{A} B})\]
\end{defn}
\begin{proof}
  $\Pi_{A}B$ acts on objects by taking fiberwise sections
  \[ \Pi_{A}B (x) := \{s \in [A (x), \Si_{A}B (x)] \st \fst_x \circ s = \id_{A (x)}\} \]
  Where we have taken the full subcategory of the functor category $[A (x), \Si_{A}B (x)]$.
  This is a groupoid since any natural transformation
  of functors into groupoids are natural isomorphisms.

  $\Pi_{A}B$ acts on morphisms via conjugation
  % https://q.uiver.app/#q=WzAsMTAsWzAsMCwieCJdLFswLDIsInkiXSxbMSwxXSxbMiwxXSxbMywwLCJcXFBpX0EgQiAoeCkiXSxbMywyLCJcXFBpX0EgQih5KSJdLFs2LDAsIkEgKHgpIl0sWzgsMCwiXFxTaV97QX1CICh4KSJdLFs2LDIsIkEgKHkpIl0sWzgsMiwiXFxTaV97QX1CICh5KSJdLFswLDEsImYiXSxbMiwzLCJcXFBpX0EgQiIsMCx7InN0eWxlIjp7InRhaWwiOnsibmFtZSI6Im1hcHMgdG8ifX19XSxbNCw1LCJcXFNpX0FCKGYpIFxcY2lyYyAtIFxcY2lyYyBBIChmXnstMX0pIl0sWzgsNiwiQShmXnstMX0pIiwxXSxbOCw5LCJcXFBpX0EgQiAoZikgKHMpIiwyLHsic3R5bGUiOnsiYm9keSI6eyJuYW1lIjoiZGFzaGVkIn19fV0sWzYsNywicyIsMV0sWzcsOSwiXFxTaV9BIEIoZikiLDFdXQ==
  \[\begin{tikzcd}
    x &&& {\Pi_A B (x)} &&& {A (x)} && {\Si_{A}B (x)} \\
    & {} & {} \\
    y &&& {\Pi_A B(y)} &&& {A (y)} && {\Si_{A}B (y)}
    \arrow["f", from=1-1, to=3-1]
    \arrow["{\Si_AB(f) \circ - \circ A (f^{-1})}", from=1-4, to=3-4]
    \arrow["s"{description}, from=1-7, to=1-9]
    \arrow["{\Si_A B(f)}"{description}, from=1-9, to=3-9]
    \arrow["{\Pi_A B}", maps to, from=2-2, to=2-3]
    \arrow["{A(f^{-1})}"{description}, from=3-7, to=1-7]
    \arrow["{\Pi_A B (f) (s)}"', dashed, from=3-7, to=3-9]
  \end{tikzcd}\]
  Note that conjugation is functorial and invertible.
\end{proof}

\medskip

\begin{prop}[All objects are fibrant]
  Let $\terminal$ denote the terminal groupoid,
  namely that with a single object and morphism.
  Then the unique map $\Ga \to \terminal$ is a fibration.
\end{prop}

\medskip

\begin{defn}[Interval]
  Let the interval groupoid $\Interval$ be the small groupoid with two objects
  and a single non-identity isomorphism.
  There are two distinct morphisms $\de_{0}, \de_{1} : \terminal \to \Interval$
  and a natural isomorphism $i : \de_{0} \Rightarrow \de_{1}$.
  Note that $\de_{0}$ and $\de_{1}$ both form adjoint equivalences with the unique
  map $! : \Interval \to \terminal$.

  Denote by $\Two$ the small groupoid with two objects and only identity morphisms.
  Then let $\partial : \Two \to \Interval$ be the unique map factoring $\de_{0}$ and $\de_{1}$.

  % https://q.uiver.app/#q=WzAsNCxbMSwxLCJcXFR3byJdLFsyLDIsIlxcSW50ZXJ2YWwiXSxbMCwxLCJcXHRlcm1pbmFsIl0sWzEsMCwiXFx0ZXJtaW5hbCJdLFswLDEsIlxccGFydGlhbCIsMSx7InN0eWxlIjp7ImJvZHkiOnsibmFtZSI6ImRhc2hlZCJ9fX1dLFsyLDEsIlxcZGVfMCIsMix7ImN1cnZlIjoxfV0sWzMsMSwiXFxkZV8xIiwwLHsiY3VydmUiOi0xfV0sWzIsMF0sWzMsMF1d
  \[\begin{tikzcd}
    & \terminal \\
    \terminal & \Two \\
    && \Interval
    \arrow[from=1-2, to=2-2]
    \arrow["{\de_1}", bend left, from=1-2, to=3-3]
    \arrow[from=2-1, to=2-2]
    \arrow["{\de_0}"', bend right, from=2-1, to=3-3]
    \arrow["\partial"{description}, dashed, from=2-2, to=3-3]
  \end{tikzcd}\]
\end{defn}

\medskip

\begin{prop}[Path object fibration]
  Let $\Ga$ be a small groupoid.
  Recall that $\grpd$ is Cartesian closed,
  so we can take the image of the above diagram under the functor $\Ga^{-}$.
  % https://q.uiver.app/#q=WzAsNCxbMSwxLCJcXEdhIFxcdGltZXMgXFxHYSJdLFsxLDIsIlxcR2EiXSxbMiwxLCJcXEdhIl0sWzAsMCwiXFxHYV5cXEludGVydmFsIl0sWzAsMV0sWzAsMl0sWzMsMSwiXFxHYV57XFxkZV8wfSIsMix7ImN1cnZlIjoxLCJzdHlsZSI6eyJoZWFkIjp7Im5hbWUiOiJlcGkifX19XSxbMywyLCJcXEdhXntcXGRlXzF9IiwwLHsiY3VydmUiOi0xLCJzdHlsZSI6eyJoZWFkIjp7Im5hbWUiOiJlcGkifX19XSxbMywwLCJcXEdhXlxccGFydGlhbCIsMSx7InN0eWxlIjp7ImJvZHkiOnsibmFtZSI6ImRhc2hlZCJ9LCJoZWFkIjp7Im5hbWUiOiJlcGkifX19XV0=
  \[\begin{tikzcd}
    {\Ga^\Interval} \\
    & {\Ga \times \Ga} & \Ga \\
    & \Ga
    \arrow["{\Ga^\partial}"{description}, dashed, two heads, from=1-1, to=2-2]
    \arrow["{\Ga^{\de_1}}", bend left, two heads, from=1-1, to=2-3]
    \arrow["{\Ga^{\de_0}}"', bend right, two heads, from=1-1, to=3-2]
    \arrow[from=2-2, to=2-3]
    \arrow[from=2-2, to=3-2]
  \end{tikzcd}\]
  Then the indicated morphisms are fibrations,
  and $\Ga^{\de_{0}}, \Ga^{\de_{1}}$ form adjoint equivalences
  with $\Ga^{!} : \Ga \to \Ga^{\Interval}$.
\end{prop}

\medskip

\subsection{Polynomial endofunctors}

\begin{defn}[Polynomial endofunctor on a morphism
  in an locally Cartesian closed category]
  Let $\catC$ be a locally Cartesian closed category
  (we will take presheaves on small groupoids).
  This means for each morphism $t : B \to A$ we have an adjoint triple
  % https://q.uiver.app/#q=WzAsMyxbMCwyXSxbMSwyLCJcXGNhdEMgLyBBIl0sWzEsMCwiXFxjYXRDIC8gQiJdLFsyLDEsInRfKiIsMCx7Im9mZnNldCI6LTV9XSxbMiwxLCJ0XyEiLDIseyJvZmZzZXQiOjV9XSxbMSwyLCJ0XioiLDFdLFs1LDMsIiIsMix7ImxldmVsIjoxLCJzdHlsZSI6eyJuYW1lIjoiYWRqdW5jdGlvbiJ9fV0sWzQsNSwiIiwyLHsibGV2ZWwiOjEsInN0eWxlIjp7Im5hbWUiOiJhZGp1bmN0aW9uIn19XV0=
  \[\begin{tikzcd}
    & {\catC / B} \\
    \\
    {} & {\catC / A}
    \arrow[""{name=0, anchor=center, inner sep=0}, "{t_*}", bend left, shift left=5, from=1-2, to=3-2]
    \arrow[""{name=1, anchor=center, inner sep=0}, "{t_!}"', bend right, shift right=5, from=1-2, to=3-2]
    \arrow[""{name=2, anchor=center, inner sep=0}, "{t^*}"{description}, from=3-2, to=1-2]
    \arrow["\dashv"{anchor=center}, draw=none, from=1, to=2]
    \arrow["\dashv"{anchor=center}, draw=none, from=2, to=0]
  \end{tikzcd}\]
  where $t^{*}$ is pullback, and $t_{!}$ is composition with $t$.

  Let $t : B \to A$ be a morphism in $\catC$.
  Then define $\Poly{t} : \catC \to \catC$ be the composition
  \[
    \Poly{t} := A_{!} \circ t_{*} \circ B^{*}
    \quad \quad \quad
    % https://q.uiver.app/#q=WzAsNCxbMCwwLCJcXGNhdEMiXSxbMSwwLCJcXGNhdEMgLyBCIl0sWzIsMCwiXFxjYXRDIC8gQSJdLFszLDAsIlxcY2F0QyJdLFswLDEsIkJeKiJdLFsxLDIsInRfKiJdLFsyLDMsIkFfISJdXQ==
    \begin{tikzcd}
    \catC & {\catC / B} & {\catC / A} & \catC
    \arrow["{B^*}", from=1-1, to=1-2]
    \arrow["{t_*}", from=1-2, to=1-3]
    \arrow["{A_!}", from=1-3, to=1-4]
  \end{tikzcd}\]
\end{defn}

\medskip

\begin{prop}
  The data of a map into the polynomial applied to an object in $\catC$
  % https://q.uiver.app/#q=WzAsMixbMCwwLCJYIl0sWzEsMCwiXFxQb2x5e3R9IFkiXSxbMCwxXV0=
  \[\begin{tikzcd}
    X & {\Poly{t} Y}
    \arrow["\phi", from=1-1, to=1-2]
  \end{tikzcd}\]
  corresponds to
  % https://q.uiver.app/#q=WzAsMyxbMCwwLCJYIl0sWzIsMCwiXFxQb2x5e3R9IFkiXSxbMSwxLCJBIl0sWzAsMSwiXFxwaGkiXSxbMSwyLCJ0XyogQl4qIFkiXSxbMCwyLCJcXGFsIiwyLHsic3R5bGUiOnsiYm9keSI6eyJuYW1lIjoiZGFzaGVkIn19fV1d
  \[\begin{tikzcd}
    X && {\Poly{t} Y} \\
    & A
    \arrow["\phi", from=1-1, to=1-3]
    \arrow["\al"', dashed, from=1-1, to=2-2]
    \arrow["{t_* B^* Y}", from=1-3, to=2-2]
  \end{tikzcd}\]
  Applying the adjunction $A_{!} \dashv A^{*}$, this corresponds to
  % https://q.uiver.app/#q=WzAsMyxbMCwwLCJCXyEgdF4qXFxhbCJdLFsyLDAsIkIgXFx0aW1lcyBZIl0sWzEsMSwiQiJdLFswLDEsIlxcdGlsZGV7XFxwaGl9IiwwLHsic3R5bGUiOnsiYm9keSI6eyJuYW1lIjoiZGFzaGVkIn19fV0sWzEsMiwiQl4qIFkiXSxbMCwyLCJ0XipcXGFsIiwyXV0=
  \[
    \al : X \to A
    \quad \quad \text{ and }
    \quad \quad
    \begin{tikzcd}
    {B_! t^*\al} && {B \times Y} \\
    & B
    \arrow["{\tilde{\phi}}", dashed, from=1-1, to=1-3]
    \arrow["{t^*\al}"', from=1-1, to=2-2]
    \arrow["{B^* Y}", from=1-3, to=2-2]
  \end{tikzcd}\]
  Applying the adjunction $t^{*} \dashv t_{*}$,
  this corresponds to
  % https://q.uiver.app/#q=WzAsMixbMCwwLCJCXyEgdF4qXFxhbCJdLFsyLDAsIlkiXSxbMCwxLCJcXHRpbGRle1xcdGlsZGV7XFxwaGl9fSIsMCx7InN0eWxlIjp7ImJvZHkiOnsibmFtZSI6ImRhc2hlZCJ9fX1dXQ==
  \[
    \al : X \to A
    \quad \quad \text{ and }
    \quad \quad
    \begin{tikzcd}
    {B_! t^*\al} && Y
    \arrow["{\beta}", dashed, from=1-1, to=1-3]
  \end{tikzcd}\]
\end{prop}

\medskip

\begin{prop}[$\Poly{\tp}\Type$ classifies dependent types]
  Specialized to $\tp : \Term \to \Type$ in $\Pshgrpd$,
  the previous proposition says that a map
  from a representable
  $\Ga \to \Poly{\tp}\Type$
  corresponds to the data of
  \[
    A : \Ga \to \Type
    \quad \quad
    \text{ and }
    \quad \quad
    B : \Ga \cdot A \to \Type
  \]
  which by Yoneda corresponds to the data in $\Cat$ of
  \[
    A : \Ga \to \grpd
    \quad \quad
    \text{ and }
    \quad \quad
    B : \Ga \cdot A \to \grpd
  \]

\end{prop}

\medskip

\subsection{$\Pi$ and $\Si$ structure}

\medskip

\begin{defn}[Interpretation of $\Pi$ and $\la$]
  We define the natural transformation
  $\Pi : \Poly{\tp} \Type \to \Type$
  by first taking some small groupoid $\Ga$ and defining
  \[\Pi_{\Ga} : \Pshgrpd(\Ga, \Poly{\tp} \Type) \to \Pshgrpd(\Ga, \Type)\]
  Let $(A,B) : \Ga \to \Poly{\tp} \Type$, corresponding to $A : \Ga \to \grpd$
  and $B : \Ga \cdot A \to \grpd$.
  Taking the pushforward of fibrations in $\grpd$
  (formally defined as operations on the classifying maps),
  we obtain $\Pi_{A}B : \Ga \to \grpd$ corresponding by Yoneda
  to an element of $\Pshgrpd(\Ga, \Type)$.

  As indicated in the diagram, we take this to be the pushforward of the
  dependent display map $\disp{B}$ along the display map it depends on $\disp{A}$.
  Note that this pushforward is in $\grpd$,
  and this pushforward is only defined on fibrations.

  TODO: define $\la$.
\end{defn}
\begin{proof}
  TODO: naturality.
  % Beck-chevalley.
  % Steve says the coherence problem is not an issue here -
  % i.e. that our equalities are strict.

  TODO: prove pullback.
\end{proof}

\medskip

\begin{defn}[Interpretation of $\Si$]
  Sketch: we define the natural transformation
  $\Si : \Poly{\tp} \Type \to \Type$
  by first taking some small groupoid $\Ga$ and defining
  \[\Si_{\Ga} : \Pshgrpd(\Ga, \Poly{\tp} \Type) \to \Pshgrpd(\Ga, \Type)\]
  Again,
  this amounts to taking a pair of composable groupoid fibrations
  to a single groupoid fibration on the codomain
  % https://q.uiver.app/#q=WzAsNixbMCwxLCJcXEdhIFxcY2RvdCBBIl0sWzAsMCwiXFxHYSBcXGNkb3QgQSBcXGNkb3QgQiJdLFsxLDEsIlxcR2EiXSxbMywwLCJcXEdhIFxcY2RvdCBcXFNpX0EgQiJdLFszLDEsIlxcR2EiXSxbMiwwLCJcXG1hcHN0byJdLFsxLDAsIlxcZGlzcHtCfSIsMix7InN0eWxlIjp7ImhlYWQiOnsibmFtZSI6ImVwaSJ9fX1dLFswLDIsIlxcZGlzcHtBfSIsMix7InN0eWxlIjp7ImhlYWQiOnsibmFtZSI6ImVwaSJ9fX1dLFszLDQsIihcXGRpc3B7QX0pXyEgXFxkaXNwe0J9IiwwLHsic3R5bGUiOnsiaGVhZCI6eyJuYW1lIjoiZXBpIn19fV1d
  \[\begin{tikzcd}
    {\Ga \cdot A \cdot B} && \mapsto & {\Ga \cdot \Si_A B} \\
    {\Ga \cdot A} & \Ga && \Ga
    \arrow["{\disp{B}}"', two heads, from=1-1, to=2-1]
    \arrow["{(\disp{A})_! \disp{B}}", two heads, from=1-4, to=2-4]
    \arrow["{\disp{A}}"', two heads, from=2-1, to=2-2]
  \end{tikzcd}\]
  As indicated in the diagram, we take this to be the composition of
  $\disp{B}$ and $\disp{A}$,
  recalling that fibrations are closed under composition.

  TODO: define $\pair$.
\end{defn}
\begin{proof}
  TODO: naturality.
  % Beck-chevalley again.

  TODO: prove pullback.
\end{proof}
