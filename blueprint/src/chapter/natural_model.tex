In this chapter we describe the categorical semantics of
our syntax via natural models.
It follows previous work on natural models \cite{awodey2017naturalmodelshomotopytype},
with the following additional features
\begin{enumerate}
  \item A more compact description of identity types %TODO reference (Garner?)
    exploiting the technology of polynomial endofunctors.
  \item A collection of N Russell-style nested universes.
  \item universe-variable $\Pi$-types and $\Si$-types,
    i.e.\ with possibly different universe level inputs,
    and landing in the largest universe
    (imitating the type theory of \texttt{Lean4}).
\end{enumerate}

\medskip

\section{Interpretation of syntax}

A very brief overview of the interpretation of syntax follows.
We work in a presheaf category $\pshC$.
A context $\Ga$ is interpreted as an object $\IntpCtx{\Ga} \in \catC$.
We often take the image of the context under the Yoneda embedding
$\yo \IntpCtx{\Ga} \in \pshC$.
If $i \le N$ is a universe level, then
a typing judgment $\Ga \vdash_i a : A$ is interpreted as
a commuting triangle of the following form
\begin{center}
\begin{tikzcd}
	& {\Term_i} \\
	{\yo \IntpCtx{\Ga}} & {\Type_i}
	\arrow["{\tp_i}", from=1-2, to=2-2]
	\arrow["{\IntpTerm{a}}", from=2-1, to=1-2]
	\arrow["{\IntpType{A}}"', from=2-1, to=2-2]
\end{tikzcd}
\end{center}

\section{Natural model}

Fix a small category $\catC$.

\medskip

\begin{defn}[Natural model]
  \label{defn:NaturalModel}
  \lean{CategoryTheory.NaturalModel.NaturalModelBase} \leanok
  Following Awodey \cite{awodey2017naturalmodelshomotopytype},
  we say that a map $\tp : \Term \to \Type$ in $\pshC$ is
  \emph{fiberwise representable}
  or a \emph{natural model} when
  every fiber is representable.
  In other words, given any $\Ga \in \catC$ and any map $A : \yo (\Ga) \to \Type$,
  there is some representable $\Ga \cdot A \in \catC$ and maps $\disp{A} : \Ga \cdot A \to \Ga$
  and $\var_{A} : \yo (\Ga \cdot A) \to \Term$ forming a pullback
% https://q.uiver.app/#q=WzAsNCxbMCwwLCJcXHlvIChcXEdhIFxcY2RvdCBBKSJdLFswLDEsIlxceW8gKFxcR2EpIl0sWzEsMSwiXFxUeXBlIl0sWzEsMCwiXFxUZXJtIl0sWzAsMSwiXFx5byAoXFxkaXNwe0F9KSIsMl0sWzEsMiwiQSIsMl0sWzMsMiwiXFx0cCJdLFswLDMsIlxcdmFyX0EiXSxbMCwyLCIiLDEseyJzdHlsZSI6eyJuYW1lIjoiY29ybmVyIn19XV0=
\begin{center}
	\begin{tikzcd}
	{\yo (\Ga \cdot A)} & \Term \\
	{\yo (\Ga)} & \Type
	\arrow["{\var_A}", from=1-1, to=1-2]
	\arrow["{\yo (\disp{A})}"', from=1-1, to=2-1]
	\arrow["\lrcorner"{anchor=center, pos=0.125}, draw=none, from=1-1, to=2-2]
	\arrow["\tp", from=1-2, to=2-2]
	\arrow["A"', from=2-1, to=2-2]
\end{tikzcd}
\end{center}
\end{defn}

\medskip

\begin{defn}[Russell universes]
  \label{defn:NaturalModelRussell}
  \uses{defn:NaturalModel}
  % \lean{CategoryTheory.NaturalModel.NaturalModelRussell}
  A collection of $N+1$ natural models with
  $N$ Russell style universes and
  lifts consists of
  \begin{itemize}
    \item For each $i \leq N$ a natural model $\tp_i : \Term_i \to \Type_i$
    \item For each $i < N$ a lift $\Lift{i}{i+1} : \Type_i \to \Type_{i+1}$
    \item For each $i < N$ a point $\U_i : 1 \to \Type_{i+1}$ such that
    \begin{center}
    % https://q.uiver.app/#q=WzAsNSxbMywyLCJcXFR5cGVfe2krMX0iXSxbMSwyLCIxIl0sWzEsMCwiMSBcXGNkb3QgXFxVX2kiXSxbMywwLCJcXFRlcm1fe2krMX0iXSxbMCwwLCJcXFR5cGVfe2l9Il0sWzIsMV0sWzEsMCwiXFxVX2kiLDJdLFszLDAsIlxcdHBfe2krMX0iXSxbMiwzXSxbMiwwLCIiLDEseyJzdHlsZSI6eyJuYW1lIjoiY29ybmVyIn19XSxbNCwyLCJcXGlzbyIsMSx7InN0eWxlIjp7ImJvZHkiOnsibmFtZSI6Im5vbmUifSwiaGVhZCI6eyJuYW1lIjoibm9uZSJ9fX1dXQ==
    \begin{tikzcd}
      {\Type_{i}} & {1 \cdot \U_i} && {\Term_{i+1}} \\
      \\
      & 1 && {\Type_{i+1}}
      \arrow["\iso"{description}, draw=none, from=1-1, to=1-2]
      \arrow[from=1-2, to=1-4]
      \arrow[from=1-2, to=3-2]
      \arrow["\lrcorner"{anchor=center, pos=0.125}, draw=none, from=1-2, to=3-4]
      \arrow["{\tp_{i+1}}", from=1-4, to=3-4]
      \arrow["{\U_i}"', from=3-2, to=3-4]
    \end{tikzcd}
    \end{center}
    \end{itemize}
\end{defn}

\section{Product types}

\begin{defn}\label{defn:NaturalModelPi}
% \lean{CategoryTheory.NaturalModel.NaturalModelRussellPi}
\uses{defn:NaturalModelRussell, defn:UvPoly}
We will use $\Poly{\tp_i}$ to denote the polynomial endofunctor
\ref{defn:UvPoly} associated with a natural model $\tp_i$.
Then additional structure of $\Pi$ types on our $N$ universes consists of,
for each $i, j \le N$, a pullback square
\begin{center}
% https://q.uiver.app/#q=WzAsNCxbMCwwLCJcXFBvbHl7XFx0cF9pfXtcXFRlcm1fan0iXSxbMCwxLCJcXFBvbHl7XFx0cF9pfXtcXFR5cGVfan0iXSxbMSwwLCJcXFRlcm1fe1xcbWF4KGksail9Il0sWzEsMSwiXFxUeXBlX3tcXG1heChpLGopfSJdLFswLDEsIlxcUG9seXtcXHRwX2l9e1xcdHBfan0iLDJdLFswLDIsIlxcbGEiXSxbMiwzLCJcXHRwX3tcXG1heChpLGopfSJdLFsxLDMsIlxcUGkiLDJdLFswLDMsIiIsMSx7InN0eWxlIjp7Im5hbWUiOiJjb3JuZXIifX1dXQ==
\begin{tikzcd}
	{\Poly{\tp_i}{\Term_j}} & {\Term_{\max(i,j)}} \\
	{\Poly{\tp_i}{\Type_j}} & {\Type_{\max(i,j)}}
	\arrow["\la", from=1-1, to=1-2]
	\arrow["{\Poly{\tp_i}{\tp_j}}"', from=1-1, to=2-1]
	\arrow["\lrcorner"{anchor=center, pos=0.125}, draw=none, from=1-1, to=2-2]
	\arrow["{\tp_{\max(i,j)}}", from=1-2, to=2-2]
	\arrow["\Pi"', from=2-1, to=2-2]
\end{tikzcd}
\begin{equation}
  \label{eq:PiPullback}
\end{equation}
\end{center}
\end{defn}

% TODO: details, interpretation of syntax

\section{Sum types}

\begin{defn}\label{defn:NaturalModelSigma}
% \lean{CategoryTheory.NaturalModel.NaturalModelSigma} \leanok
\uses{defn:NaturalModel, defn:NaturalModelRussell}

We will use the polynomial composition of two maps
\ref{defn:PolynomialComposition},
$\tp_i \PolyComp \tp_j : Q \to \Poly{\tp_i}(\Type_j)$.
Then additional structure of $\Si$ types on our $N$ universes consists of,
for each $i, j \le N$, a pullback square
\begin{center}
% https://q.uiver.app/#q=WzAsNCxbMCwwLCJRIl0sWzAsMSwiXFxQb2x5e1xcdHBfaX17XFxUeXBlX2p9Il0sWzEsMCwiXFxUZXJtX3tcXG1heChpLGopfSJdLFsxLDEsIlxcVHlwZV97XFxtYXgoaSxqKX0iXSxbMCwxLCJcXHRwX2kgXFxQb2x5Q29tcCBcXHRwX2oiLDJdLFswLDIsIlxccGFpciJdLFsyLDMsIlxcdHBfe1xcbWF4KGksail9Il0sWzEsMywiXFxTaSIsMl0sWzAsMywiIiwxLHsic3R5bGUiOnsibmFtZSI6ImNvcm5lciJ9fV1d
\begin{tikzcd}
	Q & {\Term_{\max(i,j)}} \\
	{\Poly{\tp_i}{\Type_j}} & {\Type_{\max(i,j)}}
	\arrow["\pair", from=1-1, to=1-2]
	\arrow["{\tp_i \PolyComp \tp_j}"', from=1-1, to=2-1]
	\arrow["\lrcorner"{anchor=center, pos=0.125}, draw=none, from=1-1, to=2-2]
	\arrow["{\tp_{\max(i,j)}}", from=1-2, to=2-2]
	\arrow["\Si"', from=2-1, to=2-2]
\end{tikzcd}
\begin{equation}
    \label{eq:SiPullback}
\end{equation}
\end{center}
\end{defn}

% % TODO: details, interpretation of syntax

\section{Identity types}

\begin{defn}\label{defn:NaturalModelId}
\lean{CategoryTheory.NaturalModel.NaturalModelId} \leanok
\uses{defn:NaturalModel, defn:UvPolySlice}
Suppose $\tp : \Term \to \Type$ is a natural model and
we have a commutative square (this need not be a pullback)
% https://q.uiver.app/#q=WzAsNCxbMCwwLCJcXFRlcm0iXSxbMCwxLCJcXHRwIFxcdGltZXNfXFxUeXBlIFxcdHAiXSxbMSwxLCJcXFR5cGUiXSxbMSwwLCJcXFRlcm0iXSxbMCwxLCJcXGRlIiwyXSxbMSwyLCJcXElkIiwyXSxbMywyLCJcXHRwIl0sWzAsMywiXFxyZWZsIl1d
\begin{center}
\begin{tikzcd}
	\Term & \Term \\
	{\tp \times_\Type \tp} & \Type
	\arrow["\refl", from=1-1, to=1-2]
	\arrow["\de"', from=1-1, to=2-1]
	\arrow["\tp", from=1-2, to=2-2]
	\arrow["\Id"', from=2-1, to=2-2]
\end{tikzcd}
\begin{equation}
  \label{eq:Id}
\end{equation}
\end{center}

where $\de$ is the diagonal:
% https://q.uiver.app/#q=WzAsNSxbMCwwLCJcXFRlcm0iXSxbMSwxLCJcXHRwIFxcdGltZXNfXFxUeXBlIFxcdHAiXSxbMiwxLCJcXFRlcm0iXSxbMiwyLCJcXFR5cGUiXSxbMSwyLCJcXFRlcm0iXSxbMCwxLCJcXGRlIiwyXSxbMCwyXSxbMSwyXSxbMiwzLCJcXHRwIl0sWzQsMywiXFx0cCIsMl0sWzEsNF0sWzEsMywiIiwyLHsic3R5bGUiOnsibmFtZSI6ImNvcm5lciJ9fV0sWzAsNCwiIiwwLHsibGV2ZWwiOjIsInN0eWxlIjp7ImhlYWQiOnsibmFtZSI6Im5vbmUifX19XV0=
\begin{center}
\begin{tikzcd}
	\Term \\
	& {\tp \times_\Type \tp} & \Term \\
	& \Term & \Type
	\arrow["\de"', from=1-1, to=2-2]
	\arrow[Rightarrow, no head, bend left, from=1-1, to=2-3]
	\arrow[Rightarrow, no head, bend right, from=1-1, to=3-2]
	\arrow[from=2-2, to=2-3]
	\arrow[from=2-2, to=3-2]
	\arrow["\lrcorner"{anchor=center, pos=0.125}, draw=none, from=2-2, to=3-3]
	\arrow["\tp", from=2-3, to=3-3]
	\arrow["\tp"', from=3-2, to=3-3]
\end{tikzcd}
\end{center}

Then let $I$ be the pullback.
We get a comparison map $\rho$
% https://q.uiver.app/#q=WzAsNSxbMCwwLCJcXFRlcm0iXSxbMSwyLCJcXHRwIFxcdGltZXNfXFxUeXBlIFxcdHAiXSxbMiwyLCJcXFR5cGUiXSxbMiwxLCJcXFRlcm0iXSxbMSwxLCJJIl0sWzAsMSwiXFxkZSIsMix7ImN1cnZlIjoyfV0sWzEsMiwiXFxJZCIsMl0sWzMsMiwiXFx0cCJdLFswLDMsIlxccmVmbCIsMCx7ImN1cnZlIjotM31dLFs0LDFdLFs0LDNdLFs0LDIsIiIsMix7InN0eWxlIjp7Im5hbWUiOiJjb3JuZXIifX1dLFswLDQsIlxccmhvIiwxLHsic3R5bGUiOnsiYm9keSI6eyJuYW1lIjoiZGFzaGVkIn19fV1d
\begin{center}
\begin{tikzcd}
	\Term \\
	& I & \Term \\
	& {\tp \times_\Type \tp} & \Type
	\arrow["\rho"{description}, dashed, from=1-1, to=2-2]
	\arrow["\refl", bend left, from=1-1, to=2-3]
	\arrow["\de", bend right, from=1-1, to=3-2]
	\arrow[from=2-2, to=2-3]
	\arrow[from=2-2, to=3-2]
	\arrow["\lrcorner"{anchor=center, pos=0.125}, draw=none, from=2-2, to=3-3]
	\arrow["\tp", from=2-3, to=3-3]
	\arrow["\Id"', from=3-2, to=3-3]
\end{tikzcd}
\end{center}

Then view $\rho : \tp \to q$ as a map in the slice over $\Type$.
% https://q.uiver.app/#q=WzAsNSxbMCwwLCJcXFRlcm0iXSxbMSwyLCJcXHRwIFxcdGltZXNfXFxUeXBlIFxcdHAiXSxbMSwxLCJJIl0sWzEsMywiXFxUZXJtIl0sWzEsNCwiXFxUeXBlIl0sWzAsMSwiXFxkZSIsMSx7ImN1cnZlIjoxfV0sWzIsMV0sWzAsMiwiXFxyaG8iLDEseyJzdHlsZSI6eyJib2R5Ijp7Im5hbWUiOiJkYXNoZWQifX19XSxbMSwzLCJcXGZzdCIsMl0sWzMsNF0sWzAsNCwiXFx0cCIsMix7ImN1cnZlIjozfV0sWzAsMywiIiwyLHsiY3VydmUiOjIsImxldmVsIjoyLCJzdHlsZSI6eyJoZWFkIjp7Im5hbWUiOiJub25lIn19fV0sWzIsNCwicSIsMCx7ImN1cnZlIjotNX1dXQ==
\begin{center}
\begin{tikzcd}
	\Term \\
	& I \\
	& {\tp \times_\Type \tp} \\
	& \Term \\
	& \Type
	\arrow["\rho"{description}, dashed, from=1-1, to=2-2]
	\arrow["\de"{description}, bend right = 10, from=1-1, to=3-2]
	\arrow[bend right = 25, Rightarrow, no head, from=1-1, to=4-2]
	\arrow["\tp"', bend right = 40, from=1-1, to=5-2]
	\arrow[from=2-2, to=3-2]
	\arrow["q", bend left = 50, from=2-2, to=5-2]
	\arrow["\fst"', from=3-2, to=4-2]
	\arrow[from=4-2, to=5-2]
\end{tikzcd}
\end{center}

Now (by \ref{defn:UvPolySlice})
applying $\Poly{-} : (\pshC / \Type) ^{\op} \to [\pshC, \pshC]$ to $\rho : \tp \to q$
gives us a naturality square (this also need not be a pullback).
% https://q.uiver.app/#q=WzAsNCxbMCwwLCJcXFBvbHl7cX0ge1xcVGVybX0iXSxbMCwxLCJcXFBvbHl7cX0ge1xcVHlwZX0iXSxbMSwxLCJcXFBvbHl7XFx0cH0ge1xcVGVybX0iXSxbMSwwLCJcXFBvbHl7XFx0cH0ge1xcVGVybX0iXSxbMCwxLCJcXFBvbHl7cX0ge1xcdHB9IiwyXSxbMSwyLCJcXFN0YXJ7XFxyaG99X3tcXFR5cGV9IiwyXSxbMywyLCJcXFBvbHl7XFx0cH0ge1xcdHB9Il0sWzAsMywiXFxTdGFye1xccmhvfV97XFxUZXJtfSJdXQ==
\begin{center}
\begin{tikzcd}
	{\Poly{q} {\Term}} & {\Poly{\tp} {\Term}} \\
	{\Poly{q} {\Type}} & {\Poly{\tp} {\Type}}
	\arrow["{\Star{\rho}_{\Term}}", from=1-1, to=1-2]
	\arrow["{\Poly{q} {\tp}}"', from=1-1, to=2-1]
	\arrow["{\Poly{\tp} {\tp}}", from=1-2, to=2-2]
	\arrow["{\Star{\rho}_{\Type}}"', from=2-1, to=2-2]
\end{tikzcd}
\begin{equation}
    \label{eq:JDefinition1}
\end{equation}
\end{center}
Taking the pullback $T$ and the comparison map $\ep$ we have
% https://q.uiver.app/#q=WzAsNSxbMCwwLCJcXFBvbHl7cX0ge1xcVGVybX0iXSxbMSwyLCJcXFBvbHl7cX0ge1xcVHlwZX0iXSxbMiwyLCJcXFBvbHl7XFx0cH0ge1xcVHlwZX0iXSxbMiwxLCJcXFBvbHl7XFx0cH0ge1xcVGVybX0iXSxbMSwxLCJUIl0sWzAsMSwiXFxQb2x5e3F9IHtcXHRwfSIsMix7ImN1cnZlIjoyfV0sWzEsMiwiXFxTdGFye1xccmhvfV97XFxUeXBlfSIsMl0sWzMsMiwiXFxQb2x5e1xcdHB9IHtcXHRwfSJdLFswLDMsIlxcU3RhcntcXHJob31fe1xcVGVybX0iLDAseyJjdXJ2ZSI6LTN9XSxbNCwxXSxbNCwzXSxbNCwyLCIiLDIseyJzdHlsZSI6eyJuYW1lIjoiY29ybmVyIn19XSxbMCw0LCJcXGVwIiwxLHsic3R5bGUiOnsiYm9keSI6eyJuYW1lIjoiZGFzaGVkIn19fV1d
\begin{center}
\begin{tikzcd}
	{\Poly{q} {\Term}} \\
	& T & {\Poly{\tp} {\Term}} \\
	& {\Poly{q} {\Type}} & {\Poly{\tp} {\Type}}
	\arrow["\ep"{description}, dashed, from=1-1, to=2-2]
	\arrow["{\Star{\rho}_{\Term}}", bend left, from=1-1, to=2-3]
	\arrow["{\Poly{q} {\tp}}"', bend right, from=1-1, to=3-2]
	\arrow[from=2-2, to=2-3]
	\arrow[from=2-2, to=3-2]
	\arrow["\lrcorner"{anchor=center, pos=0.125}, draw=none, from=2-2, to=3-3]
	\arrow["{\Poly{\tp} {\tp}}", from=2-3, to=3-3]
	\arrow["{\Star{\rho}_{\Type}}"', from=3-2, to=3-3]
\end{tikzcd}
\begin{equation}
  	\label{eq:JDefinition2}
\end{equation}
\end{center}
Then a natural model $\tp$ with identity types consists of a commutative square
\ref{eq:Id}, with a section $\J : T \to \Poly{q}{\Term}$ of $\ep$.
\end{defn}

\section{Binary products and Exponentials}

It is convenient to specialize $\Si$ and $\Pi$ types to their non-dependent
counterparts $\times$ and $\Exp$.

\begin{defn}[Products and exponentials]
  \label{defn:ProductsExponentials}
  \uses{defn:NaturalModelPi, defn:NaturalModelSigma, prop:UvPoly.equiv}
In the natural model we can
construct these by considering first the map

\[ (\fst, \snd) : \Type_i \times \Type_j \to \Poly{\tp_i}{\Type_j}\]

defined using the characterising property of polynomials
\ref{prop:UvPoly.equiv},
which we can visualize in

\begin{center}
% https://q.uiver.app/#q=WzAsNSxbMiwwLCJcXFRlcm1faSJdLFsyLDEsIlxcVHlwZV9pIl0sWzEsMSwiXFxUeXBlX2kgXFx0aW1lcyBcXFR5cGVfaiJdLFsxLDAsIlxcVGVybV9pIFxcdGltZXMgXFxUeXBlX2oiXSxbMCwwLCJcXFR5cGVfaiJdLFswLDEsIlxcdHBfaSJdLFsyLDEsIlxcZnN0IiwyXSxbMywyLCJcXGZzdF4qIFxcdHBfaSIsMl0sWzMsMF0sWzMsMSwiIiwxLHsic3R5bGUiOnsibmFtZSI6ImNvcm5lciJ9fV0sWzMsNCwiXFxzbmQiLDJdXQ==
\begin{tikzcd}
	{\Type_j} & {\Term_i \times \Type_j} & {\Term_i} \\
	& {\Type_i \times \Type_j} & {\Type_i}
	\arrow["\snd"', from=1-2, to=1-1]
	\arrow[from=1-2, to=1-3]
	\arrow["{\fst^* \tp_i}"', from=1-2, to=2-2]
	\arrow["\lrcorner"{anchor=center, pos=0.125}, draw=none, from=1-2, to=2-3]
	\arrow["{\tp_i}", from=1-3, to=2-3]
	\arrow["\fst"', from=2-2, to=2-3]
\end{tikzcd}
\end{center}

Then, respectively, the pullback of the diagrams \ref{eq:PiPullback}
and \ref{eq:SiPullback} for interpreting $\Pi$ and $\Si$
rules along this map give us pullback diagrams for interpreting
function types and product types.
(We simplify the situation to where $i = j$.)
% https://q.uiver.app/#q=WzAsNixbMSwwLCJcXFBvbHl7XFx0cH17XFxUZXJtfSJdLFsxLDEsIlxcUG9seXtcXHRwfXtcXFR5cGV9Il0sWzIsMCwiXFxUZXJtIl0sWzIsMSwiXFxUeXBlIl0sWzAsMSwiXFxUeXBlIFxcdGltZXMgXFxUeXBlIl0sWzAsMCwiRiJdLFswLDEsIlxcUG9seXtcXHRwfXtcXHRwfSIsMl0sWzAsMiwiXFxsYSJdLFsyLDMsIlxcdHAiXSxbMSwzLCJcXFBpIiwyXSxbMCwzLCIiLDEseyJzdHlsZSI6eyJuYW1lIjoiY29ybmVyIn19XSxbNCwxLCIoXFxmc3QsXFxzbmQpIiwyXSxbNSw0LCIoXFxkb20sXFxjb2QpIiwyXSxbNSwwLCIoXFxkb20sXFxmdW4pIiwyXSxbNCwzLCJcXEV4cCIsMix7ImN1cnZlIjo1fV0sWzUsMiwiXFxsYSIsMCx7ImN1cnZlIjotNH1dLFs1LDEsIiIsMix7InN0eWxlIjp7Im5hbWUiOiJjb3JuZXIifX1dXQ==
\begin{center}\begin{tikzcd}[column sep = large]
	F & {\Poly{\tp}{\Term}} & \Term \\
	{\Type \times \Type} & {\Poly{\tp}{\Type}} & \Type
	\arrow["{(\dom,\fun)}", from=1-1, to=1-2]
	\arrow["\la", bend left, from=1-1, to=1-3]
	\arrow["{(\dom,:d)}"', from=1-1, to=2-1]
	\arrow["\la", from=1-2, to=1-3]
	\arrow["{\Poly{\tp}{\tp}}"', from=1-2, to=2-2]
	\arrow["\lrcorner"{anchor=center, pos=0.125}, draw=none, from=1-1, to=2-2]
	\arrow["\lrcorner"{anchor=center, pos=0.125}, draw=none, from=1-2, to=2-3]
	\arrow["\tp", from=1-3, to=2-3]
	\arrow["{(\fst,\snd)}"', from=2-1, to=2-2]
	\arrow["\Exp"', bend right, from=2-1, to=2-3]
	\arrow["\Pi"', from=2-2, to=2-3]
\end{tikzcd}\end{center}

% https://q.uiver.app/#q=WzAsNixbMSwwLCJRIl0sWzEsMSwiXFxQb2x5e1xcdHB9e1xcVHlwZX0iXSxbMiwwLCJcXFRlcm0iXSxbMiwxLCJcXFR5cGUiXSxbMCwxLCJcXFR5cGUgXFx0aW1lcyBcXFR5cGUiXSxbMCwwLCJcXFRtIFxcdGltZXMgXFxUbSJdLFswLDEsIlxcdHAgXFx0cmlhbmdsZWxlZnQgXFx0cCIsMl0sWzAsMiwiXFxwYWlyIl0sWzIsMywiXFx0cCJdLFsxLDMsIlxcU2kiLDJdLFswLDMsIiIsMSx7InN0eWxlIjp7Im5hbWUiOiJjb3JuZXIifX1dLFs0LDEsIihcXGZzdCxcXHNuZCkiLDJdLFs1LDQsIlxcdHAgXFx0aW1lcyBcXHRwIiwyXSxbNSwwLCIoXFxzbmQsIFxcZnN0LCBcXHRwIFxcY2lyYyBcXHNuZCkiXSxbNCwzLCJcXHRpbWVzIiwyLHsiY3VydmUiOjV9XSxbNSwyLCJcXHBhaXIiLDAseyJjdXJ2ZSI6LTR9XSxbNSwxLCIiLDAseyJzdHlsZSI6eyJuYW1lIjoiY29ybmVyIn19XV0=
\begin{center}\begin{tikzcd}[column sep = large]
	{\Term \times \Term} & Q & \Term \\
	{\Type \times \Type} & {\Poly{\tp}{\Type}} & \Type
	\arrow["{(\snd, \fst, \tp \circ \snd)}", from=1-1, to=1-2]
	\arrow["\pair", bend left, from=1-1, to=1-3]
	\arrow["{\tp \times \tp}"', from=1-1, to=2-1]
	\arrow["\lrcorner"{anchor=center, pos=0.125}, draw=none, from=1-1, to=2-2]
	\arrow["\pair", from=1-2, to=1-3]
	\arrow["{\tp \PolyComp \tp}"', from=1-2, to=2-2]
	\arrow["\lrcorner"{anchor=center, pos=0.125}, draw=none, from=1-2, to=2-3]
	\arrow["\tp", from=1-3, to=2-3]
	\arrow["{(\fst,\snd)}"', from=2-1, to=2-2]
	\arrow["\times"', bend right, from=2-1, to=2-3]
	\arrow["\Si"', from=2-2, to=2-3]
\end{tikzcd}\end{center}
\end{defn}


By the universal property of pullbacks and
\ref{prop:UvPoly.equiv}
we can write a map $\Ga \to F$ as a triple $(A,B,f)$
such that $A, B: \Ga \to \Type$ and
% https://q.uiver.app/#q=WzAsNCxbMCwwLCJcXEdhIFxcY2RvdCBBIl0sWzEsMCwiXFxUZXJtIl0sWzEsMSwiXFxUeXBlIl0sWzAsMSwiXFxHYSJdLFswLDEsImYiXSxbMSwyLCJcXHRwIl0sWzMsMiwiQiIsMl0sWzAsMywiXFxkaXNwe0F9IiwyXV0=
\begin{center}\begin{tikzcd}
	{\Ga \cdot A} & \Term \\
	\Ga & \Type
	\arrow["f", from=1-1, to=1-2]
	\arrow["{\disp{A}}"', from=1-1, to=2-1]
	\arrow["\tp", from=1-2, to=2-2]
	\arrow["B"', from=2-1, to=2-2]
\end{tikzcd}\end{center}

This gives us four equivalent ways we can view a function.
Namely, as $f : \Ga \cdot A \to \Term$ in the above diagram,
$\la \circ f : \Ga \to \Term$,
as $(A,B,f) : \Ga \to F$, or as a map between the displays
$\disp{A} \to \disp{B}$

% https://q.uiver.app/#q=WzAsNSxbMSwyLCJcXEdhIl0sWzAsMCwiXFxHYSBcXGNkb3QgQSJdLFsxLDEsIlxcR2EgXFxjZG90IEIiXSxbMiwxLCJcXFRlcm0iXSxbMiwyLCJcXFR5cGUiXSxbMSwwLCJcXGRpc3B7QX0iLDJdLFsyLDAsIlxcZGlzcHtCfSIsMV0sWzEsMiwiKFxcZGlzcHtBfSxmKSIsMSx7InN0eWxlIjp7ImJvZHkiOnsibmFtZSI6ImRhc2hlZCJ9fX1dLFsyLDNdLFszLDRdLFswLDQsIkIiLDJdLFsyLDQsIiIsMix7InN0eWxlIjp7Im5hbWUiOiJjb3JuZXIifX1dLFsxLDMsImYiXV0=
\begin{center}\begin{tikzcd}
	{\Ga \cdot A} \\
	& {\Ga \cdot B} & \Term \\
	& \Ga & \Type
	\arrow["{(\disp{A},f)}"{description}, dashed, from=1-1, to=2-2]
	\arrow["f", bend left, from=1-1, to=2-3]
	\arrow["{\disp{A}}"', bend right, from=1-1, to=3-2]
	\arrow[from=2-2, to=2-3]
	\arrow["{\disp{B}}"{description}, from=2-2, to=3-2]
	\arrow["\lrcorner"{anchor=center, pos=0.125}, draw=none, from=2-2, to=3-3]
	\arrow[from=2-3, to=3-3]
	\arrow["B"', from=3-2, to=3-3]
\end{tikzcd}\end{center}

For the formalization, we need not prove that the pullback of
$\tp \PolyComp \tp$ is $\tp \times \tp$.
Rather, we can also use the universal property of pullbacks
and \ref{prop:UvPoly.equiv}
to classify a map into the pullback (whatever it may be) as a
pair $(\al, \be)$, where $\al, \be : \Ga \to \Term$.
This could then be adapted to a proof that the pullback is
what the diagram claims it to be.

\begin{defn}
  \label{defn:IdComp}
  \uses{defn:ProductsExponentials}
The identity function $\id_{A} : \Ga \to \Term$ of type $\Exp \circ (A,A) : \Ga \to \Type$
can be defined by the following

% https://q.uiver.app/#q=WzAsOSxbMCwyLCJcXEdhIl0sWzAsMSwiXFxHYSBcXGNkb3QgQSJdLFsxLDEsIlxcVGVybSJdLFsxLDIsIlxcVHlwZSJdLFszLDAsIlxcR2EiXSxbNCwxLCJGIl0sWzUsMSwiXFxUZXJtIl0sWzUsMiwiXFxUeXBlIl0sWzQsMiwiXFxUeXBlIFxcdGltZXMgXFxUeXBlIl0sWzEsMCwiXFxkaXNwe0F9IiwxXSxbMSwyLCJcXHZhcl97QX0iLDJdLFsyLDMsIlxcdHAiXSxbMCwzLCJBIiwyXSxbMSwzLCIiLDIseyJzdHlsZSI6eyJuYW1lIjoiY29ybmVyIn19XSxbNCw1LCIoQSxBLFxcdmFyX3tBfSkiLDFdLFs1LDYsIlxcbGEiXSxbNiw3LCJcXHRwIl0sWzgsNywiXFxFeHAiLDJdLFs1LDgsIihcXGRvbSwgXFxjb2QpIiwxXSxbNCw4LCIoQSxBKSIsMl0sWzQsNiwiXFxpZF9BIiwwLHsic3R5bGUiOnsiYm9keSI6eyJuYW1lIjoiZGFzaGVkIn19fV1d
\begin{center}\begin{tikzcd}
	&&& \Ga \\
	{\Ga \cdot A} & \Term &&& F & \Term \\
	\Ga & \Type &&& {\Type \times \Type} & \Type
	\arrow["{(A,A,\var_{A})}"{description}, from=1-4, to=2-5]
	\arrow["{\id_A}", bend left, dashed, from=1-4, to=2-6]
	\arrow["{(A,A)}"', bend right, from=1-4, to=3-5]
	\arrow["{\var_{A}}"', from=2-1, to=2-2]
	\arrow["{\disp{A}}"{description}, from=2-1, to=3-1]
	\arrow["\lrcorner"{anchor=center, pos=0.125}, draw=none, from=2-1, to=3-2]
	\arrow["\tp", from=2-2, to=3-2]
	\arrow["\la", from=2-5, to=2-6]
	\arrow["{(\dom, :d)}"{description}, from=2-5, to=3-5]
	\arrow["\tp", from=2-6, to=3-6]
	\arrow["A"', from=3-1, to=3-2]
	\arrow["\Exp"', from=3-5, to=3-6]
\end{tikzcd}\end{center}

Viewed as a map between the display maps, this is simply the identity
$\Ga \cdot A \to \Ga \cdot A$.
% https://q.uiver.app/#q=WzAsNSxbMSwyLCJcXEdhIl0sWzEsMSwiXFxHYSBcXGNkb3QgQSJdLFsyLDEsIlxcVGVybSJdLFsyLDIsIlxcVHlwZSJdLFswLDAsIlxcR2EgXFxjZG90IEEiXSxbMSwwLCJcXGRpc3B7QX0iLDFdLFsxLDIsIlxcdmFyX3tBfSIsMl0sWzIsMywiXFx0cCJdLFswLDMsIkEiLDJdLFsxLDMsIiIsMix7InN0eWxlIjp7Im5hbWUiOiJjb3JuZXIifX1dLFs0LDAsIlxcZGlzcHtBfSIsMl0sWzQsMSwiIiwxLHsibGV2ZWwiOjIsInN0eWxlIjp7ImJvZHkiOnsibmFtZSI6ImRhc2hlZCJ9LCJoZWFkIjp7Im5hbWUiOiJub25lIn19fV0sWzQsMiwiXFx2YXJfQSJdXQ==
\begin{center}\begin{tikzcd}
	{\Ga \cdot A} \\
	& {\Ga \cdot A} & \Term \\
	& \Ga & \Type
	\arrow[Rightarrow, dashed, no head, from=1-1, to=2-2]
	\arrow["{\var_A}", bend left, from=1-1, to=2-3]
	\arrow["{\disp{A}}"', bend right, from=1-1, to=3-2]
	\arrow["{\var_{A}}"', from=2-2, to=2-3]
	\arrow["{\disp{A}}"{description}, from=2-2, to=3-2]
	\arrow["\lrcorner"{anchor=center, pos=0.125}, draw=none, from=2-2, to=3-3]
	\arrow["\tp", from=2-3, to=3-3]
	\arrow["A"', from=3-2, to=3-3]
\end{tikzcd}\end{center}

Composition is also simplest when viewed as an operation on maps between fibers.
Given $f : \disp{A} \to \disp{B}$ and $g : \disp{B} \to \disp{C}$,
the composition is $g \circ f : \disp{A} \to \disp{C}$.
\end{defn}

\section{Univalence}

For two types $A, B : \Ga \to \Type$
and two functions $f, g : A \to B$ we can define internally
a \textit{homotopy} from $f$ to $g$ as
\[ f \sim g := \Pi_{a : A} \, \Id (f \, a, g \, a) \]
We define the types of left and right inverses of $f : A \to B$ as
\[ \mathsf{BigLinv} f := \Si_{g : B \to A} \, g \circ f \sim \id_{A} \]
\[ \mathsf{BigRinv} f := \Si_{g : B \to A} \, f \circ g \sim \id_{B} \]
and the property of being an equivalence
\[ \mathsf{IsBigEquiv} f := \mathsf{BigLinv} f \times \mathsf{BigRinv} f \]

We could do the same for two small types $A, B : \Ga \to \U$

\[ \mathsf{IsEquiv} f := \mathsf{Linv} f \times \mathsf{Rinv} f \]
\[ \mathsf{Equiv} A \, B := \Si_{f : A \to B} \mathsf{IsEquiv} f\]

Again, internally we can define a function
\[ \mathsf{IdToEquiv} A \, B : \Id (A, B) \to \mathsf{Equiv} A \, B \]
which uses $J$ to transport along the proof of equality to produce an equivalence.

\begin{defn}\label{defn:NaturalModelUnivalentU}
\lean{CategoryTheory.NaturalModel.NaturalModelUnivalentU}
\notready
\uses{defn:NaturalModelRussell}
  Univalence for universe $\U$ states that \textsf{IdToEquiv} itself is an equivalence
  \[ \mathsf{ua} : \mathsf{IsBigEquiv} (\mathsf{IdToEquiv} A \, B)\]
  Note that this statement is large, i.e. not a type in the universe $\U$.
  % https://q.uiver.app/#q=WzAsMyxbMSwxLCJcXFUgXFxjZG90IFxcVSJdLFswLDAsIlxcVSBcXGNkb3QgXFxVIFxcY2RvdCBcXElkIl0sWzIsMCwiXFxVIFxcY2RvdCBcXFUgXFxjZG90IFxcbWF0aHNme0VxdWl2fSJdLFsxLDBdLFsyLDBdLFsxLDIsIlxcbWF0aHNme0lkVG9FcXVpdn0iXV0=
  \begin{center}\begin{tikzcd}
	          {\U \cdot \U \cdot \Id} && {\U \cdot \U \cdot \mathsf{Equiv}} \\
	          & {\U \cdot \U}
	          \arrow["{\mathsf{IdToEquiv}}", from=1-1, to=1-3]
	          \arrow[from=1-1, to=2-2]
	          \arrow[from=1-3, to=2-2]
  \end{tikzcd}\end{center}

\end{defn}

\section{Extensional identity types and UIP}

In this section we outline variations on the identity type in the
natural model.
We will describe these as additional structure on $\Id$,
as opposed to introducing different identity types.

\medskip

\begin{defn}[Extensional types]
  \label{defn:NaturalModelEq}
  \uses{defn:NaturalModel}
The first option is fully extensional identity types,
i.e. those satisfying equality reflection and uniqueness of identity proofs (UIP).
Equality reflection says that if one can construct a term satisfying $\Id(a,b)$
then we have that definitionally $a \equiv b$,
i.e. they are equal morphisms in the natural model.
This amounts to just requiring that \ref{eq:Id} is a pullback,
i.e. $\rho$ is an isomorphism
% https://q.uiver.app/#q=WzAsNCxbMCwwLCJcXFRlcm0iXSxbMCwxLCJcXHRwIFxcdGltZXNfXFxUeXBlIFxcdHAiXSxbMSwxLCJcXFR5cGUiXSxbMSwwLCJcXFRlcm0iXSxbMCwxLCJcXGRlIiwyXSxbMSwyLCJcXElkIiwyXSxbMywyLCJcXHRwIl0sWzAsMywiXFxyZWZsIl0sWzAsMiwiIiwxLHsic3R5bGUiOnsibmFtZSI6ImNvcm5lciJ9fV1d
\begin{center}\begin{tikzcd}
	\Term & \Term \\
	{\tp \times_\Type \tp} & \Type
	\arrow["\refl", from=1-1, to=1-2]
	\arrow["\de"', from=1-1, to=2-1]
	\arrow["\lrcorner"{anchor=center, pos=0.125}, draw=none, from=1-1, to=2-2]
	\arrow["\tp", from=1-2, to=2-2]
	\arrow["\Id"', from=2-1, to=2-2]
\end{tikzcd}\end{center}
Note that this means $\Star{\rho}$ is an isomorphism,
from which it follows that \ref{eq:JDefinition1} is also a pullback,
i.e. $\ep$ is an isomorphism.
% https://q.uiver.app/#q=WzAsNCxbMCwwLCJcXFBvbHl7cX0ge1xcVGVybX0iXSxbMCwxLCJcXFBvbHl7cX0ge1xcVHlwZX0iXSxbMSwxLCJcXFBvbHl7XFx0cH0ge1xcVGVybX0iXSxbMSwwLCJcXFBvbHl7XFx0cH0ge1xcVGVybX0iXSxbMCwxLCJcXFBvbHl7cX0ge1xcdHB9IiwyXSxbMSwyLCJcXFN0YXJ7XFxyaG99X3tcXFR5cGV9IiwyXSxbMywyLCJcXFBvbHl7XFx0cH0ge1xcdHB9Il0sWzAsMywiXFxTdGFye1xccmhvfV97XFxUZXJtfSJdLFswLDIsIiIsMSx7InN0eWxlIjp7Im5hbWUiOiJjb3JuZXIifX1dXQ==
\begin{center}\begin{tikzcd}
	{\Poly{q} {\Term}} & {\Poly{\tp} {\Term}} \\
	{\Poly{q} {\Type}} & {\Poly{\tp} {\Term}}
	\arrow["{\Star{\rho}_{\Term}}", from=1-1, to=1-2]
	\arrow["{\Poly{q} {\tp}}"', from=1-1, to=2-1]
	\arrow["\lrcorner"{anchor=center, pos=0.125}, draw=none, from=1-1, to=2-2]
	\arrow["{\Poly{\tp} {\tp}}", from=1-2, to=2-2]
	\arrow["{\Star{\rho}_{\Type}}"', from=2-1, to=2-2]
\end{tikzcd}\end{center}
\end{defn}

\medskip

We could only require UIP:

\medskip

\begin{defn}[Identity types satisfying UIP]
  \label{defn:NaturalModelIdUIP}
  \uses{defn:NaturalModelId}
  Say an identity type in a natural model satisfies UIP if
  $I \to \tp \times_{\Type} \tp$ is a strict proposition,
  meaning for any $(a,b) : \Ga \to \tp \times_{\Type} \tp$
  there it at most one lift
% https://q.uiver.app/#q=WzAsMyxbMCwxLCJcXEdhIl0sWzEsMSwiXFx0cCBcXHRpbWVzX1xcVHlwZSBcXHRwIl0sWzEsMCwiSSJdLFswLDEsIihhLGIpIiwyXSxbMiwxXSxbMCwyLCIhIiwwLHsic3R5bGUiOnsiYm9keSI6eyJuYW1lIjoiZGFzaGVkIn19fV1d
\begin{center}\begin{tikzcd}
	& I \\
	\Ga & {\tp \times_\Type \tp}
	\arrow[from=1-2, to=2-2]
	\arrow["{!}", dashed, from=2-1, to=1-2]
	\arrow["{(a,b)}"', from=2-1, to=2-2]
\end{tikzcd}\end{center}
\end{defn}

One might wonder what other variations we could
come up with by tweaking the pullback conditions.
In fact, only requiring that $\rho$ has a section
is equivalent to requiring that $\rho$ is an isomorphism.
So this just the extensional case again.

If we require instead that $\ep$ is an isomorphism
then this is giving an $\eta$-rule for $J$,
from which we can prove equality reflection and UIP
\cite{streicher1995ExtensionalConceptsInIntensionalTypeTheory}.
So this just the extensional case again.
