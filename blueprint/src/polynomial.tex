In this section we develop some of the definitions
and lemmas related to polynomial endofunctors that we will use
in the rest of the notes.

\begin{defn}[Polynomial endofunctor]
  Let $\catC$ be a locally Cartesian closed category
  (in our case, presheaves on the category of contexts).
  This means for each morphism $t : B \to A$ we have an adjoint triple
  % https://q.uiver.app/#q=WzAsMyxbMCwyXSxbMSwyLCJcXGNhdEMgLyBBIl0sWzEsMCwiXFxjYXRDIC8gQiJdLFsyLDEsInRfKiIsMCx7Im9mZnNldCI6LTV9XSxbMiwxLCJ0XyEiLDIseyJvZmZzZXQiOjV9XSxbMSwyLCJ0XioiLDFdLFs1LDMsIiIsMix7ImxldmVsIjoxLCJzdHlsZSI6eyJuYW1lIjoiYWRqdW5jdGlvbiJ9fV0sWzQsNSwiIiwyLHsibGV2ZWwiOjEsInN0eWxlIjp7Im5hbWUiOiJhZGp1bmN0aW9uIn19XV0=
  \[\begin{tikzcd}
    & {\catC / B} \\
    \\
    {} & {\catC / A}
    \arrow[""{name=0, anchor=center, inner sep=0}, "{t_*}", bend left, shift left=5, from=1-2, to=3-2]
    \arrow[""{name=1, anchor=center, inner sep=0}, "{t_!}"', bend right, shift right=5, from=1-2, to=3-2]
    \arrow[""{name=2, anchor=center, inner sep=0}, "{t^*}"{description}, from=3-2, to=1-2]
    \arrow["\dashv"{anchor=center}, draw=none, from=1, to=2]
    \arrow["\dashv"{anchor=center}, draw=none, from=2, to=0]
  \end{tikzcd}\]
  where $t^{*}$ is pullback, and $t_{!}$ is composition with $t$.

  Let $t : B \to A$ be a morphism in $\catC$.
  Then define $\Poly{t} : \catC \to \catC$ be the composition
  \[
    \Poly{t} := A_{!} \circ t_{*} \circ B^{*}
    \quad \quad \quad
    % https://q.uiver.app/#q=WzAsNCxbMCwwLCJcXGNhdEMiXSxbMSwwLCJcXGNhdEMgLyBCIl0sWzIsMCwiXFxjYXRDIC8gQSJdLFszLDAsIlxcY2F0QyJdLFswLDEsIkJeKiJdLFsxLDIsInRfKiJdLFsyLDMsIkFfISJdXQ==
    \begin{tikzcd}
    \catC & {\catC / B} & {\catC / A} & \catC
    \arrow["{B^*}", from=1-1, to=1-2]
    \arrow["{t_*}", from=1-2, to=1-3]
    \arrow["{A_!}", from=1-3, to=1-4]
  \end{tikzcd}\]
\end{defn}

\medskip

\begin{prop}[Characterising property of Polynomial Endofunctors]
  \label{polynomial_endofunctor_property}
  The data of a map into the polynomial applied to an object in $\catC$
% https://q.uiver.app/#q=WzAsMixbMCwwLCJcXEdhIl0sWzEsMCwiXFxQb2x5e3R9IFkiXSxbMCwxXV0=
\[\begin{tikzcd}
	\Ga & {\Poly{t} Y}
	\arrow[from=1-1, to=1-2]
\end{tikzcd}\]
  corresponds to
% https://q.uiver.app/#q=WzAsMyxbMCwwLCJcXEdhIl0sWzIsMCwiXFxQb2x5e3R9IFkiXSxbMSwxLCJBIl0sWzAsMSwiXFxwaGkiXSxbMSwyLCJ0XyogQl4qIFkiXSxbMCwyLCJcXGFsIiwyLHsic3R5bGUiOnsiYm9keSI6eyJuYW1lIjoiZGFzaGVkIn19fV1d
\[\begin{tikzcd}
	\Ga && {\Poly{t} Y} \\
	& A
	\arrow["\phi", from=1-1, to=1-3]
	\arrow["\al"', dashed, from=1-1, to=2-2]
	\arrow["{t_* B^* Y}", from=1-3, to=2-2]
\end{tikzcd}\]
  Applying the adjunction $A_{!} \dashv A^{*}$, this corresponds to
  \[
    \al : \Ga \to A
    \quad \quad \text{ and }
    \quad \quad
  % https://q.uiver.app/#q=WzAsMyxbMCwwLCJCXyEgdF4qXFxhbCJdLFsyLDAsIkIgXFx0aW1lcyBZIl0sWzEsMSwiQiJdLFswLDEsIlxcdGlsZGV7XFxwaGl9IiwwLHsic3R5bGUiOnsiYm9keSI6eyJuYW1lIjoiZGFzaGVkIn19fV0sWzEsMiwiQl4qIFkiXSxbMCwyLCJ0XipcXGFsIiwyXV0=
    \begin{tikzcd}
    {B_! t^*\al} && {B \times Y} \\
    & B
    \arrow["{\tilde{\phi}}", dashed, from=1-1, to=1-3]
    \arrow["{t^*\al}"', from=1-1, to=2-2]
    \arrow["{B^* Y}", from=1-3, to=2-2]
  \end{tikzcd}
\]
  Applying the adjunction $t^{*} \dashv t_{*}$,
  this corresponds to
  % https://q.uiver.app/#q=WzAsMixbMCwwLCJCXyEgdF4qXFxhbCJdLFsyLDAsIlkiXSxbMCwxLCJcXHRpbGRle1xcdGlsZGV7XFxwaGl9fSIsMCx7InN0eWxlIjp7ImJvZHkiOnsibmFtZSI6ImRhc2hlZCJ9fX1dXQ==
  \[
    \al : \Ga \to A
    \quad \quad \text{ and }
    \quad \quad
    \begin{tikzcd}
    {B_! t^*\al} && Y
    \arrow["{\beta}", dashed, from=1-1, to=1-3]
  \end{tikzcd}\]

  Henceforth we will write
  \[ (\al, \be) : \Ga \to \Poly{t} Y\]
  for this map,
  since it is uniquely determined by this data.
  We also have that for any $\si : \De \to \Ga$,
  the following triangle commutes
% https://q.uiver.app/#q=WzAsNCxbMiwxLCJcXFBvbHl7dH0gWSJdLFswLDFdLFsxLDEsIlxcR2EiXSxbMSwwLCJcXERlIl0sWzMsMiwiXFxzaSIsMl0sWzMsMCwiKFxcYWwgXFxjaXJjIFxcc2ksIFxcYmUgXFxjaXJjIHReKiBcXHNpKSJdLFsyLDAsIihcXGFsLCBcXGJlKSIsMl1d
\[\begin{tikzcd}
	& \De \\
	{} & \Ga & {\Poly{t} Y}
	\arrow["\si"', from=1-2, to=2-2]
	\arrow["{(\al \circ \si, \be \circ t^* \si)}", from=1-2, to=2-3]
	\arrow["{(\al, \be)}"', from=2-2, to=2-3]
\end{tikzcd}\]
\end{prop}

\medskip
