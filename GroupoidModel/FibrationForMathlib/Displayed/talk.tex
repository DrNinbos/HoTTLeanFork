% Don't forget to use 'xdvi -paper usr' to get landscape in xdvi
% and to use 'dvipdfm -l' to get the landscape pdf
\documentclass[landscape]{slides}
\usepackage{enumerate, amsfonts, amsmath, url, graphicx, color,nopageno}
\usepackage{pifont}
\usepackage[T1]{fontenc}
\usepackage[landscape]{geometry}
\usepackage{amsthm, MnSymbol}
\usepackage{hyperref}
\usepackage[utf8x]{inputenc}
\usepackage{upgreek}
\usepackage{xcolor}
\usepackage{colortbl}
\usepackage[shortlabels,inline]{enumitem}
\usepackage[capitalise]{cleveref}
\usepackage{subcaption}
\usepackage{tikz-cd}
\usepackage{tikz}
\usepackage{tabularx}
\usepackage{listings}
\usepackage{flushend}
\usepackage{xspace}
\usepackage{underscore}
\usepackage{microtype}
\usepackage{scalefnt}
\usepackage{macro}
%
\usepackage{fancyvrb}
\usepackage{multicol}

% \usepackage{fontspec}
% \usepackage{unicode-math}
% \setmainfont{DejaVu Serif}
% \setsansfont{DejaVu Sans}
% \setmathfont{XITS Math}
% \setmonofont{DejaVu Sans Mono}

\newtheorem*{definition}{Definition}
\newtheorem*{theorem}{Theorem}
\newtheorem*{lemma}{Lemma}

\definecolor{keywordcolor}{rgb}{0.7, 0.1, 0.1}   % red
\definecolor{tacticcolor}{rgb}{0.0, 0.1, 0.3}    % dark blue
\definecolor{commentcolor}{rgb}{0.4, 0.4, 0.4}   % grey
\definecolor{stringcolor}{rgb}{0.5, 0.3, 0.2}    % brown
\definecolor{symbolcolor}{rgb}{0.1, 0.2, 0.7}    % blue
\definecolor{sortcolor}{rgb}{0.1, 0.5, 0.1}      % green
\definecolor{attributecolor}{rgb}{0.7, 0.1, 0.1} % red
\definecolor{errorcolor}{rgb}{1, 0, 0}           % bright red

\def\lstlanguagefiles{lstlean.tex}
\lstloadlanguages{lean}
\lstset{language=lean,belowskip=-2em, fancyvrb=true}

\DeclareMathOperator{\Hom}{Hom}

\newcommand{\N}{\mathbb{N}}
\newcommand{\Z}{\mathbb{Z}}
\newcommand{\Q}{\mathbb{Q}}
\newcommand{\R}{\mathbb{R}}
\newcommand{\Circ}{\mathbb{S}}
\newcommand{\mathlib}{\texttt{mathlib}\xspace}

\newcommand{\lean}[1]{{\color{red}{\texttt{#1}}}}

\definecolor{bluegray}{rgb}{0.4, 0.6, 0.8}

\begin{document}

\begin{slide}
   \begin{center}
      \textcolor{bluegray}{\Large Fibred Categories in Lean}
      \par\vskip\baselineskip
       Sina Hazratpour\\
      \par\vskip\baselineskip
      16 Apr, 2024
   \end{center}
\end{slide}

\begin{slide}
  \restoregeometry
  % Use below to fiddle with margins (useful for displayed Lean)
  % \newgeometry{left=3cm,right=3cm,left=2cm,right=2cm}
  Here is how inline Lean looks:
  \par\vskip0.5\baselineskip
  \begin{lstlisting}
import Mathlib

def IsLimit (f : ℝ → ℝ) (x y : ℝ) : Prop :=
  ∀ ε > 0, ∃ δ > 0, ∀ x',
    dist x' x < δ → dist (f x') y < ε    
  \end{lstlisting}

  I sometimes mess with the separation parameters for bulleted lists like this:
  \begin{itemize}[itemsep=0cm,topsep=-0.5cm]
    \item Foo
    \item Bar
  \end{itemize}
\end{slide}


%%%%%%%%%%%%%%%%%%%%%%%%%%%%%%%%%%%%%%%%%%%%%%%%%%%%%%%%%%%%
%%%%%%%%%%%%%%%%%%%%%%%%%%%%%%%%%%%%%%%%%%%%%%%%%%%%%%%%%%%%
\begin{slide}
The notion of Fibred categories is an important tool which allows us to talk about several important concepts and constructions in category theory. 

% and they have important applications in algebraic geometry, categorical logic, models of dependent type theories, categorical algebra. 

They appear in 

\begin{itemize}[itemsep=0cm,topsep=-0.5cm]
 \item Algebraic geometry (e.g. stacks) 
 \item Models of type theories (comprehension categories, fibred notion of structures, refinement types, etc.)
 \item Internal category theory (externalization functor and comprehension construction)
 \item Relative category theory (relative topoi)
 \item Categorical Deep Learning (parametric lenses)
 \item Grothendieck/Dialectica/Chu construction 
 \item \ldots
\end{itemize}  
\end{slide}


%%%%%%%%%%%%%%%%%%%%%%%%%%%%%%%%%%%%%%%%%%%%%%%%%%%%%%%%%%%%
%%%%%%%%%%%%%%%%%%%%%%%%%%%%%%%%%%%%%%%%%%%%%%%%%%%%%%%%%%%%
\begin{slide}
  \restoregeometry
  %
  \begin{definition}
    A functor $\PP : \EE \to \CC$ is a \textbf{Grothendieck fibration} if for each each object $X$ in $\EE$ and each 
    morphism $f \colon J \to P X$ in $\CC$ there exists a {cartesian} lift $\widehat{f}$ of $f$ in $\EE$ with the codomain $X$. 
  \end{definition}
  %
  \[
  \begin{tikzpicture}[scale=1]
    \node (cloud-rt) at (14.5,10.5) {\tikz \asymcloud[.25]{fill=gray!8,thick};};
    \node (cloud-rb) at (14.5,5) {\tikz \asymcloud[.23]{fill=gray!20};}; 
     \node (mid-l) at (7.9,9) {};
     \node (mid-r) at (11.9,9) {};
     \node (mid-down-l) at (7.9,4.5) {};
     \node (mid-down-r) at (11.9,4.5) {};
     \node (rc-mid-t) at (14.4,8.1) {};
     \node (rc-mid-b) at (14.4,6.5) {};
     \node (rca) at (13.5,10.1) {\scriptsize$\widehat{J}$};
     \node (rcc) at (15.5,10.1) {\scriptsize$X$};
     \node (rc-top) at (8.2,12) {$\EE$};
     \node (rc-bot) at (8.2,3.0) {$\CC$};
     \node (rbca) at (13.5,4.6) {\scriptsize$J$};
     \node (rbcb) at (15.5,4.6) {\scriptsize$I$};
     \draw[->] 
    (rbca.east) to node(e1){} node[above]{\scriptsize$f$} (rbcb.west);
    \draw[->,thick] 
    (rc-top) to node(e4){} node[left] {$\PP$} (rc-bot);
     \draw[->,thick] 
    (rc-mid-t) to[bend left = 20] node(e3){} node[right] {} (rc-mid-b);
     \draw[->, dashed] 
    (rca.east) to node(e1){} node[above] {\scriptsize$\widehat{f}$} (rcc.west);
  \end{tikzpicture}
  \] 
\end{slide}



%%%%%%%%%%%%%%%%%%%%%%%%%%%%%%%%%%%%%%%%%%%%%%%%%%%%%%%%%%%%
%% A bad naive definition 
%%%%%%%%%%%%%%%%%%%%%%%%%%%%%%%%%%%%%%%%%%%%%%%%%%%%%%%%%%%%
\begin{slide}
  \newgeometry{left=1cm,right=1cm,left=2cm,right=2cm}
\begin{lstlisting}
  
  import Mathlib
  
  variable {C E : Type*} [Category C] [Category E]

  def GrothendieckFibration (P : E ⥤ C) := 
      ∀ {J : C} {X : E} (f : J ⟶ P.obj X), 
        ∃ (Y : E) (h : P.obj Y = J) (g : Y ⟶ X), P.map g = eqToHom (h) ≫ f ∧ 
          ∀ {K : C} (u : K ⟶ J) (Z : E) (h' : P.obj Z = K) (k : Z ⟶ X), 
            (P.map k = eqToHom (h') ≫ u ≫ f) → 
              ∃! (m : Z ⟶ Y), P.map m = eqToHom (h') ≫ u ≫ eqToHom (h.symm)
  
\end{lstlisting}
  
\end{slide}


%%%%%%%%%%%%%%%%%%%%%%%%%%%%%%%%%%%%%%%%%%%%%%%%%%%%%%%%%%%%
%% We don't want this! 
%%%%%%%%%%%%%%%%%%%%%%%%%%%%%%%%%%%%%%%%%%%%%%%%%%%%%%%%%%%%
\begin{slide}
  \newgeometry{left=1cm,right=1cm,left=2cm,right=2cm}
  \begin{lstlisting}
    
    import Mathlib
    
    variable {C E : Type*} [Category C] [Category E]
  
    def GrothendieckFibration (P : E ⥤ C) := 
        ∀ {J : C} {X : E} (f : J ⟶ P.obj X), 
          ∃ (Y : E) (h : P.obj Y = J) (g : Y ⟶ X), P.map g = eqToHom (h) ≫ f ∧ 
            ∀ {K : C} (u : K ⟶ J) (Z : E) (h' : P.obj Z = K) (k : Z ⟶ X), 
              (P.map k = eqToHom (h') ≫ u ≫ f) → 
                ∃! (m : Z ⟶ Y), P.map m = eqToHom (h') ≫ u ≫ eqToHom (h.symm)
    
  \end{lstlisting}
  
  



\end{slide}



%%%%%%%%%%%%%%%%%%%%%%%%%%%%%%%%%%%%%%%%%%%%%%%%%%%%%%%%%%%%
%%%%%%%%%%%%%%%%%%%%%%%%%%%%%%%%%%%%%%%%%%%%%%%%%%%%%%%%%%%%
\begin{slide}
  \restoregeometry
  %
  \begin{definition}
    A functor $\PP : \EE \to \CC$ is a \textbf{Grothendieck fibration} if for each 
    morphism $f \colon J \to I$ in $\CC$ and each object 
    $X$ in the fibre $P^{-1}(I)$ there exists a {cartesian} lift $\widehat{f}$ of $f$ in $\EE$ with the codomain $X$. 
  \end{definition}
  %
  \[
  \begin{tikzpicture}[scale=1]
    \node (cloud-rt) at (14.5,10.5) {\tikz \asymcloud[.25]{fill=gray!8,thick};};
    \node (cloud-rb) at (14.5,5) {\tikz \asymcloud[.23]{fill=gray!20};}; 
     \node (mid-l) at (7.9,9) {};
     \node (mid-r) at (11.9,9) {};
     \node (mid-down-l) at (7.9,4.5) {};
     \node (mid-down-r) at (11.9,4.5) {};
     \node (rc-mid-t) at (14.4,8.1) {};
     \node (rc-mid-b) at (14.4,6.5) {};
     \node (rca) at (13.5,10.1) {\scriptsize$\widehat{J}$};
     \node (rcc) at (15.5,10.1) {\scriptsize$X$};
     \node (rc-top) at (8.2,12) {$\EE$};
     \node (rc-bot) at (8.2,3.0) {$\CC$};
     \node (rbca) at (13.5,4.6) {\scriptsize$J$};
     \node (rbcb) at (15.5,4.6) {\scriptsize$I$};
     \draw[->] 
    (rbca.east) to node(e1){} node[above]{\scriptsize$f$} (rbcb.west);
    \draw[->,thick] 
    (rc-top) to node(e4){} node[left] {$\PP$} (rc-bot);
     \draw[->,thick] 
    (rc-mid-t) to[bend left = 20] node(e3){} node[right] {} (rc-mid-b);
     \draw[->, dashed] 
    (rca.east) to node(e1){} node[above] {\scriptsize$\widehat{f}$} (rcc.west);
  \end{tikzpicture}
  \] 
\end{slide}





%% Note: We should formally specify  
%% 1. fibres of a functor and 
%% 2. what it means to lift a morphism from the base category to the total category


%%%%%%%%%%%%%%%%%%%%%%%%%%%%%%%%%%%%%%%%%%%%%%%%%%%%%%%%
%% formalizing fibres of a functor
%%%%%%%%%%%%%%%%%%%%%%%%%%%%%%%%%%%%%%%%%%%%%%%%%%%%%%%%
\begin{slide}
  \restoregeometry
  We need to talk about \textbf{fibres}. 
  \par\vskip1\baselineskip
  %
  \begin{lstlisting}
    import Mathlib
    
    variable {C E : Type*} [Category C] [Category E] 

    def Fibre (P : E ⥤ C) (c : C) :=
    {d : E // P.obj d = c}

    notation:75 P " ⁻¹ " c => Fibre P c

    instance {c : C} : CoeOut (P⁻¹ c) E where
      coe := fun x => x.1
  \end{lstlisting}

\end{slide}  


%%%%%%%%%%%%%%%%%%%%%%%%%%%%%%%%%%%%%%%%%%%%%%%%%%%%%%%%
%%lifts
%%%%%%%%%%%%%%%%%%%%%%%%%%%%%%%%%%%%%%%%%%%%%%%%%%%%%%%%
\begin{slide}
  \restoregeometry
  We need to talk about \textbf{lifts}: 


  A lift $\widehat{f}$ of $f$ starting at $X$ and ending at $Y$ is a diagonal filler for the following commutative diagram of functors. 

  \par\vskip1\baselineskip 
  %based-lift structure as filler 
  \[
  \begin{tikzpicture}[scale=2]
  \node (lu) at (0,2) {$\mathbb{1} + \mathbb{1}$};
  \node (ld) at (0,0) {$\mathbb{2}$};
  \node (ru) at (2.5,2) {$\EE$};
  \node (rd) at (2.5,0) {$\CC$};
  \draw[>->]
  (lu) to node(e0){} node[left] {}  (ld);
  \draw[->] 
  (lu) to node(e1){} node[above] {\small $[X,Y]$}  (ru);
  \draw[->] 
  (ru) to node(e2){} node[right] {\small $\PP$}  (rd);
  \draw[->] 
  (ld) to node(e3){} node[below] {\small $f$}  (rd);
  \draw[->,dashed]
  (ld) to node(e4){} node[above, sloped] {$\widehat{f}$} (ru);
  \end{tikzpicture}
  \]  

\end{slide}  


%%%%%%%%%%%%%%%%%%%%%%%%%%%%%%%%%%%%%%%%%%%%%%%%%%%%%%%%
%% based lifts as fillers
%%%%%%%%%%%%%%%%%%%%%%%%%%%%%%%%%%%%%%%%%%%%%%%%%%%%%%%%
\begin{slide}
    \restoregeometry 

    \textbf{Q}: Could we define based-lift structure as the type of diagonal filler of the above square and get rid of \lean{eqToHom}. 
    
    \textbf{A}: Issues with rewriting along the equality of functors in Lean makes this approach difficult. 
    
    \par\vskip0.5\baselineskip
    
    \begin{lstlisting}
      import Mathlib
    
      variable {C D E : Type*} [Category C] [Category D] [Category E] (F G : C ⥤ D) (H : D ⥤ E)
    
      example {X Y : C} (f : X ⟶ Y) (h : F ⋙ H = G ⋙ H) 
          [IsIso (H.map (F.map f))] :
          IsIso (H.map (G.map f)) := by
        rw [← Functor.comp_map]
        rw [← h] -- tactic 'rewrite' failed, motive is not type correct  

    \end{lstlisting}

\end{slide}  
  
  
%%%%%%%%%%%%%%%%%%%%%%%%%%%%%%%%%%%%%%%%%%%%%%%%%%%%%%%%
%% equality vs isomorphism 
%%%%%%%%%%%%%%%%%%%%%%%%%%%%%%%%%%%%%%%%%%%%%%%%%%%%%%%%
\begin{slide}
  \newgeometry{left=1cm,right=1cm,top=1cm,bottom=1cm}
  
    No problem if we replace equality of functors by their isomorphisms. 

    % \par\vskip1\baselineskip

    \begin{lstlisting}
   
      example {X Y : C} (f : X ⟶ Y) (h : F ⋙ H ≅ G ⋙ H) 
          [IsIso (H.map (F.map f))] :
          IsIso (H.map (G.map f)) := by
        rw [← Functor.comp_map]
        have h1 := h.hom.naturality f
        have h2 : H.map (G.map f) = h.inv.app X ≫ H.map (F.map f) 
          ≫ h.hom.app Y := by
          rw [← Functor.comp_map]; aesop_cat
        have h3 : IsIso (h.inv.app X ≫ H.map (F.map f) ≫ 
        h.hom.app Y) := by
          aesop_cat
        rw [← Functor.comp_map] at h2
        rw [← h2] at h3
        assumption

    \end{lstlisting}
  
\end{slide}  
 

%%%%%%%%%%%%%%%%%%%%%%%%%%%%%%%%%%%%%%%%%%%%%%%%%%%%%%%%
%% equality vs isomorphism 
%%%%%%%%%%%%%%%%%%%%%%%%%%%%%%%%%%%%%%%%%%%%%%%%%%%%%%%%
\begin{slide}
  \restoregeometry 

Generally, it is a good practice in category theory to replace equalities of object/functors by isomorphisms wherever we can. 

In category theory, we rarely have interesting equalities of objects.
    
Grothendieck fibrations are exception.
  
\end{slide}

%%%%%%%%%%%%%%%%%%%%%%%%%%%%%%%%%%%%%%%%%%%%%%%%%%%%%%%%
%% formalizing lifts of morphisms in the base (1)
%%%%%%%%%%%%%%%%%%%%%%%%%%%%%%%%%%%%%%%%%%%%%%%%%%%%%%%%
\begin{slide}
  \newgeometry{left=1cm,right=1cm,left=2cm,right=2cm}

  \begin{lstlisting}
    import Mathlib

    variable {C E : Type*} [Category C] [Category E] {I J : C}

    structure BasedLift (f : I ⟶ J) (X : P⁻¹ I) (Y : P⁻¹ J) 
    where
      hom : (X : E) ⟶ (Y : E)
      over : P.map hom = f  
  \end{lstlisting}
\end{slide}  


%%%%%%%%%%%%%%%%%%%%%%%%%%%%%%%%%%%%%%%%%%%%%%%%%%%%%%%%
%% formalizing lifts of morphisms in the base (2)
%%%%%%%%%%%%%%%%%%%%%%%%%%%%%%%%%%%%%%%%%%%%%%%%%%%%%%%%
\begin{slide}
  \newgeometry{left=1cm,right=1cm, top=0.5cm}

  \begin{lstlisting}
    import Mathlib

    variable {C E : Type*} [Category C] [Category E] {I J : C}

    structure BasedLift (f : I ⟶ J) (X : P⁻¹ I) (Y : P⁻¹ J) 
    where
      hom : (X : E) ⟶ (Y : E)
      over : P.map hom = f  
  \end{lstlisting}


  \par\vskip0.5\baselineskip
  {\color{red}{This specification does not type-check!}}

  \begin{center}
  \includegraphics*[width=0.5\textwidth]{images/BasedLift.png}
  \end{center}
  
\end{slide}  



%%%%%%%%%%%%%%%%%%%%%%%%%%%%%%%%%%%%%%%%%%%%%%%%%%%%%%%%
%% formalizing lifts of morphisms in the base (3)
%%%%%%%%%%%%%%%%%%%%%%%%%%%%%%%%%%%%%%%%%%%%%%%%%%%%%%%%
\begin{slide}
  % \restoregeometry
  \par\vskip0.5\baselineskip
  %
  \begin{lstlisting}
    import Mathlib

    variable {C E : Type*} [Category C] [Category E] {I J : C}
    
    structure BasedLift (f : I ⟶ J) (X : P⁻¹ I) (Y : P⁻¹ J) 
    where
      hom : (X : E) ⟶ (Y : E)
      over : (P.map hom) ≫ eqToHom (Y.2) = eqToHom (X.2) ≫ f  
  \end{lstlisting}

  
\end{slide}

%%%%%%%%%%%%%%%%%%%%%%%%%%%%%%%%%%%%%%%%%%%%%%%%%%%%%%%%
%% formalizing lifts of morphisms in the base (4)
%%%%%%%%%%%%%%%%%%%%%%%%%%%%%%%%%%%%%%%%%%%%%%%%%%%%%%%%
\begin{slide}
  \newgeometry{left=1cm,right=1cm, top=0.25cm}
  \par\vskip0.5\baselineskip
  \begin{lstlisting}
    import Mathlib
    
    variable {C E : Type*} [Category C] [Category E] {I J : C}

    structure BasedLift (f : I ⟶ J) (X : P⁻¹ I) (Y : P⁻¹ J) 
    where
      hom : (X : E) ⟶ (Y : E)
      over : (P.map hom) ≫ eqToHom (Y.2) = eqToHom (X.2) ≫ f  
  \end{lstlisting}

  \par\vskip0.25\baselineskip

  \[
  \begin{tikzpicture}[baseline=(b0l.base),scale=1,every node/.style={transform shape}]
  \node(tl) at (0,2) {$X$};
  \node(tr) at (4,2) {$Y$};
  \node(b0l) at (0,-3) {$P(X)$};
  \node(b0r) at (4,-3) {$P(Y)$};
  \node(b1l) at (0,-6) {$I$};
  \node(b1r) at (4,-6) {$J$};
  \node(tc) at (2,1) {} ;
  \node(bc) at (2,-1) {} ;
  \draw[->] (tl) to node(tar){} node[above]{$hom$} (tr);
  \draw[->] (b0l) to node(bar){} node[above]{\scriptsize$\PP (hom)$} (b0r);
  \draw[->] (b1l) to node(bar){} node[below]{$f$} (b1r);
  \draw[|->] (tc) to node(mid){} node[left]{$\PP$} (bc); 
  \draw[->,dotted] (b0l) to node(lar){} node[left]{\scriptsize  eqToHom} (b1l);
  \draw[->,dotted] (b0r) to node(rar){} node[right]{\scriptsize  eqToHom} (b1r);
  \end{tikzpicture}
  \]
\end{slide}  


%%%%%%%%%%%%%%%%%%%%%%%%%%%%%%%%%%%%%%%%%%%%%%%%%%%%%%%%
%% composition of based-lifts
%%%%%%%%%%%%%%%%%%%%%%%%%%%%%%%%%%%%%%%%%%%%%%%%%%%%%%%%
\begin{slide}
  \newgeometry{left=1cm,right=1cm, top=0.25cm}
  \par\vskip0.5\baselineskip

  We can compose lifts: 

  \par\vskip1\baselineskip

  \begin{lstlisting}
    import Mathlib

    variable {C E : Type*} [Category C] [Category E] (P : E ⥤ C)

    notation x " ⟶[" f "] " y => BasedLift (P:= _) f x y

    def comp {I J K : C} {f : I ⟶ J} {f' : J ⟶ K} 
        {X : P⁻¹ I} {Y : P⁻¹ J} {Z : P⁻¹ K} 
        (g : X ⟶[f] Y) (g' : Y ⟶[f'] Z) : X ⟶[f ≫ f'] Z :=
      ⟨g.hom ≫ g'.hom, by aesop⟩

    scoped infixr:50 "  ≫ₗ "  => BasedLift.comp  

  \end{lstlisting}

\end{slide}


%%%%%%%%%%%%%%%%%%%%%%%%%%%%%%%%%%%%%%%%%%%%%%%%%%%%%%%%
%% composition of based-lifts
%%%%%%%%%%%%%%%%%%%%%%%%%%%%%%%%%%%%%%%%%%%%%%%%%%%%%%%%
\begin{slide}
  \newgeometry{left=1cm,right=1cm, top=0.25cm}
  \par\vskip0.5\baselineskip

  But this composition is neither associative nor unital.
   
  \par\vskip1\baselineskip

  \begin{lstlisting}
    import Mathlib

    variable {C E : Type*} [Category C] [Category E] (P : E ⥤ C)

    variable {I J K L: C} {f₁ : I ⟶ J} {f₂ : J ⟶ K} {f₃ : K ⟶ L} 
             {W : P⁻¹ I} {X : P⁻¹ J} {Y : P⁻¹ K} {Z : P⁻¹ L} 

    example (g₁ : W ⟶[f₁] X) (g₂ : X ⟶[f₂] Y) (g₃ : Y ⟶[f₃] Z) :  
    ((g₁ ≫ₗ g₂) ≫ₗ g₃) = g₁ ≫ₗ g₂ ≫ₗ g₃ := by
      sorry 
  \end{lstlisting}

  returns the following error:

  \begin{lstlisting}
  type mismatch
      g₁ ≫ₗ g₂ ≫ₗ g₃
    has type
      BasedLift P (f₁ ≫ f₂ ≫ f₃) W Z : Type ?u.94482
    but is expected to have type
      BasedLift P ((f₁ ≫ f₂) ≫ f₃) W Z : Type ?u.94482
  \end{lstlisting}  

\end{slide}


%%%%%%%%%%%%%%%%%%%%%%%%%%%%%%%%%%%%%%%%%%%%%%%%%%%%%%%%
%% composition of based-lifts
%%%%%%%%%%%%%%%%%%%%%%%%%%%%%%%%%%%%%%%%%%%%%%%%%%%%%%%%
\begin{slide}
  \newgeometry{left=1cm,right=1cm, top=0.25cm}
  \par\vskip0.5\baselineskip

  We can fix this problem by casting RHS of the equality above along the associativity 
  \lean{assoc : (f₁ ≫ f₂) ≫ f₃ = f₁ ≫ (f₂ ≫ f₃)}
  in the base category $\CC$.
  
  Similar story with the unit laws of the composition for lifts. 

  We then need to develop a basic API of lifts, but need to carry around the casting terms. Not ideal! 

  Alternatively, a more abstract and principled approach is provided by the notion of \textbf{displayed categories}.

  First, we define \textbf{displayed categories} in Lean. 

\end{slide}  


%%%%%%%%%%%%%%%%%%%%%%%%%%%%%%%%%%%%%%%%%%%%%%%%%%%%%%%%
%% display categories
%%%%%%%%%%%%%%%%%%%%%%%%%%%%%%%%%%%%%%%%%%%%%%%%%%%%%%%%
\begin{slide}
  \newgeometry{left=1cm,right=1cm, top=0.25cm}
  \par\vskip0.5\baselineskip

  \begin{lstlisting}
  import Mathlib
  
  variable {C : Type*} [Category C] (F : C → Type*)

  class DisplayStruct where
  HomOver : ∀ {I J : C}, (I ⟶ J) → F I → F J → Type*
  id_over : ∀ {I : C} (X : F I), HomOver (1 I) X X
  comp_over : ∀ {I J K : C} {f₁ : I ⟶ J} {f₂ : J ⟶ K} 
              {X : F I} {Y : F J} {Z : F K}, 
              HomOver f₁ X Y → HomOver f₂ Y Z → HomOver (f₁ ≫ f₂) X Z

  notation X " ⟶[" f "] " Y => DisplayStruct.HomOver f X Y

  notation " 1ₗ " => DisplayStruct.id_over

  scoped infixr:80 "  ≫ₗ "  => DisplayStruct.comp_over

  \end{lstlisting}

\end{slide}  





\end{document}
